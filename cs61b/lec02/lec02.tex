\documentclass{scrartcl}
\usepackage[utf8]{inputenc}
\usepackage{fancyhdr}
\usepackage{graphicx}
\usepackage{dirtytalk}
\usepackage[portrait, margin=1in]{geometry}
\usepackage{amsmath, amssymb}
\setlength{\parskip}{1em}
\setlength{\parindent}{0cm} 
\usepackage[mdthm,simplethm]{test}
\usepackage[utf8]{inputenc}
\usepackage{float}
\usepackage{physics}
\setlength{\parindent}{0pt}

\title{Classes}
\author{albert ye}
\date{May 2022}

\begin{document}
\maketitle

\section{Classes}
\subsection{Compilation}
x.java $\rightarrow$ x.class $\rightarrow$ output through javac, then java

IntelliJ is secretly calling these things, so basically this means IntelliJ
doesn't need to be used in the first place. We use vim

\subsection{Generally just classes}

Executable classes need a \texttt{main} method, but that's not necessary for
classes that interact with executable classes (but aren't executable
themselves).

\subsection{Class Abstraction}
Classes can contain functions and data. Classes can also be instantiated
as objects. This saves a lot of redundancy for largely similar things with
slightly varying properties.


\texttt{static} means a shared quality, while non-static means that it is abt
a specific object. Static methods cannot access one class's instance vars,
while a non-static method can. This is why non-static elements cannot 
be used in a static context (but the other way's OK).

\texttt{this} is a reference to yourself (but you already knew that.)

\end{document}
