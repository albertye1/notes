\documentclass{book}
\usepackage{michael}

\title{Math 104: Real Analysis}
\author{Albert Ye}
\date{\today}

\begin{document}
\maketitle
\mytoc 
\chapter{Week 1}
\section{Lecture 1}
\subsection{Logic and Sets}
For clauses $p$, $q$: we have $p \land q$, $p \lor q$, $\lnot p$. These are \textit{and}, \textit{or}, \textit{not}; respectively. 

Moreover, we have $p \implies q$ meaning that $q$ is true if $p$ is true. Moreover, we have $p \iff q$ meaning that $p$ is true if $q$ is true and $q$ is true if $p$ is true.

Other terminology: $:=$ is a definition, $\forall$ is for all, $\exists$ is exists, $a \in A$ means that element $a$ is in the set $A$, $a \notin A$ means that element $a$ isn't in the set $A$.

For sets, we have $\subset, =, \subseteq$ to determine subset and equality relations. Moreover, we have $\cap, \cup$ to represent union and intersections of sets. There is also $A \setminus B$ to denote everything in $A$ but not $B$, and we have $A^C$ to denote every element not in $A$.

\begin{thm}[DeMorgan's Laws]
    Let $A$ and $B$ be sets. 
    \begin{enumerate}[label = (\alph*)]
        \item $(A \cup B)^C = A^C \cap B^C$
        \item $(A \cap B)^C = A^C \cup B^C$
        \item $A \setminus (B \cap C) = (A \setminus B) \cup (A \setminus C)$
        \item $A \setminus (B \cup C) = (A \setminus B) \cap (A \setminus C)$
    \end{enumerate}
\end{thm}

\subsection{Indexed Sets}
Let $\Lambda$ be a set and suppose for each $a \in \Lambda$ there is a set $A_a$. The set $\{A_a : a \in \Lambda\}$ is called a \vocab{collection of sets indexed by $\Lambda$}. In this case, $\Lambda$ is called the \vocab{indexing set} for this collection.

\[\bigcup_{a \in A} = \{x | x \in A_a \textrm{ for some } a \in A\}\]
\[\bigcap_{a \in A} = \{x | x \in A_a \textrm{ for all } a \in A\}.\]

We can generalize DeMorgan's laws to indexed collections: 
\begin{thm}[Generalized DeMorgan]
    If $\{B_a : a \in \Lambda\}$ is an indexed collection of sets and $A$ is a set, then 
    \[A \setminus \bigcup_{a \in \Lambda} B_a = \bigcap_{a \in \Lambda} (A \setminus B_a),\]
    \[A \setminus \bigcap_{a \in \Lambda} B_a = \bigcup_{a \in \Lambda} (A \setminus B_a).\] 
\end{thm}

\subsection{Set of Natural Numbers}
We set $\NN$ to be all positive integers, $\ZZ$ to be all integers, and $\NN_0$ to be all nonnegative integers.

\begin{definition}[Peano Axioms]
    \begin{enumerate}
        \item $1 \in \NN$.
        \item If $n \in \NN$, then $n + 1 \in \NN$. We'll call this the \vocab{successor}.
        \item $1$ is not the successor of any element
        \item If $n, m \in NN$ have the same successor, then $n = m$. 
        \item (Induction) If $S \subseteq \NN$ with the properties $1 \in S$ and $n \in S \implies n + 1 \in S$, then $S = \NN$. This becomes induction when we have $S$ as the set of elements where a certain property holds.
    \end{enumerate}
\end{definition}

So, for induction, we have a base case where we have $P_0$ or $P_1$ or some starting value. And then, we have induction that proves that $P_k$ being true implies $P_{k + 1}$ is true. Then it dominoes over.

Remember that we didn't prove that $P_{n + 1}$ is true, but rather that it can be implied from $P_n$.

\subsection{Set of Rational Numbers}
We define $\QQ$, the set of rational numbers, by $\QQ := \{\frac{m}{n} | m, n \in \ZZ, n \neq 0\}$.

\begin{remark}
    $\QQ$ contains all terminating decimals.
\end{remark}

\begin{remark}
    If $\frac{m}{n} \in \QQ$ and $r \in \ZZ \setminus \{0\}$, then $\frac{m}{n} = \frac{rm}{rn}$, so we assume that $m, n$ are coprime usually.
\end{remark}

\begin{definition}[Field Axioms]
    Remembering these is now an exercise for the reader.
\end{definition}

We see that the set of rational numbers with addition and multiplication is a field. Going through the axioms is left as an exercise to the reader.
\end{document}