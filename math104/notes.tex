\documentclass{book}
\usepackage{michael}

\title{Math 104: Real Analysis}
\author{Albert Ye}
\date{\today}

\begin{document}
\maketitle
\mytoc 
\chapter{Week 1}
\section{Lecture 1}
\subsection{Logic and Sets}
For clauses $p$, $q$: we have $p \land q$, $p \lor q$, $\lnot p$. These are \textit{and}, \textit{or}, \textit{not}; respectively. 

Moreover, we have $p \implies q$ meaning that $q$ is true if $p$ is true. Moreover, we have $p \iff q$ meaning that $p$ is true if $q$ is true and $q$ is true if $p$ is true.

Other terminology: $:=$ is a definition, $\forall$ is for all, $\exists$ is exists, $a \in A$ means that element $a$ is in the set $A$, $a \notin A$ means that element $a$ isn't in the set $A$.

For sets, we have $\subset, =, \subseteq$ to determine subset and equality relations. Moreover, we have $\cap, \cup$ to represent union and intersections of sets. There is also $A \setminus B$ to denote everything in $A$ but not $B$, and we have $A^C$ to denote every element not in $A$.

\begin{thm}[DeMorgan's Laws]
    Let $A$ and $B$ be sets. 
    \begin{enumerate}[label = (\alph*)]
        \item $(A \cup B)^C = A^C \cap B^C$
        \item $(A \cap B)^C = A^C \cup B^C$
        \item $A \setminus (B \cap C) = (A \setminus B) \cup (A \setminus C)$
        \item $A \setminus (B \cup C) = (A \setminus B) \cap (A \setminus C)$
    \end{enumerate}
\end{thm}

\subsection{Indexed Sets}
Let $\Lambda$ be a set and suppose for each $a \in \Lambda$ there is a set $A_a$. The set $\{A_a : a \in \Lambda\}$ is called a \vocab{collection of sets indexed by $\Lambda$}. In this case, $\Lambda$ is called the \vocab{indexing set} for this collection.

\[\bigcup_{a \in A} = \{x | x \in A_a \textrm{ for some } a \in A\}\]
\[\bigcap_{a \in A} = \{x | x \in A_a \textrm{ for all } a \in A\}.\]

We can generalize DeMorgan's laws to indexed collections: 
\begin{thm}[Generalized DeMorgan]
    If $\{B_a : a \in \Lambda\}$ is an indexed collection of sets and $A$ is a set, then 
    \[A \setminus \bigcup_{a \in \Lambda} B_a = \bigcap_{a \in \Lambda} (A \setminus B_a),\]
    \[A \setminus \bigcap_{a \in \Lambda} B_a = \bigcup_{a \in \Lambda} (A \setminus B_a).\] 
\end{thm}

\subsection{Set of Natural Numbers}
We set $\NN$ to be all positive integers, $\ZZ$ to be all integers, and $\NN_0$ to be all nonnegative integers.

\begin{definition}[Peano Axioms]
    \begin{enumerate}
        \item $1 \in \NN$.
        \item If $n \in \NN$, then $n + 1 \in \NN$. We'll call this the \vocab{successor}.
        \item $1$ is not the successor of any element
        \item If $n, m \in NN$ have the same successor, then $n = m$. 
        \item (Induction) If $S \subseteq \NN$ with the properties $1 \in S$ and $n \in S \implies n + 1 \in S$, then $S = \NN$. This becomes induction when we have $S$ as the set of elements where a certain property holds.
    \end{enumerate}
\end{definition}

So, for induction, we have a base case where we have $P_0$ or $P_1$ or some starting value. And then, we have induction that proves that $P_k$ being true implies $P_{k + 1}$ is true. Then it dominoes over.

Remember that we didn't prove that $P_{n + 1}$ is true, but rather that it can be implied from $P_n$.

\subsection{Set of Rational Numbers}
We define $\QQ$, the set of rational numbers, by $\QQ := \{\frac{m}{n} | m, n \in \ZZ, n \neq 0\}$.

\begin{remark}
    $\QQ$ contains all terminating decimals.
\end{remark}

\begin{remark}
    If $\frac{m}{n} \in \QQ$ and $r \in \ZZ \setminus \{0\}$, then $\frac{m}{n} = \frac{rm}{rn}$, so we assume that $m, n$ are coprime usually.
\end{remark}

\begin{definition}[Field Axioms]
    Remembering these is now an exercise for the reader.
\end{definition}

We see that the set of rational numbers with addition and multiplication is a field. Going through the axioms is left as an exercise to the reader.

\newpage
\section{Lecture 2}
\subsection{Ordered Sets}
\begin{defn}[Ordered Set]
    We define an \vocab{ordered set} to be a set $S$ with an order satisfying the following criteria:
    \begin{enumerate}
        \item $\forall \alpha, \beta \in S$, either $\alpha < \beta, \alpha = \beta, \alpha > \beta$. 
        \item $\alpha < \beta, \beta < \gamma \implies \alpha < \gamma$.
        \item $\alpha \leq \beta \implies \alpha + \gamma \leq \beta + \gamma$
        \item $\alpha \leq \beta, \gamma \geq 0 \implies \alpha \gamma \leq \beta \gamma$
        \item $\alpha \leq \beta, \beta \leq \alpha \implies \alpha = \beta$
    \end{enumerate}
\end{defn}

A set that is a field and an ordered set can be called an \vocab{ordered field}.

\subsection{Defects of $\QQ$}
\begin{thm}[Irrationality of $\sqrt 2$]
    There is no $\alpha$ such that $\alpha^2 = 2$.
\end{thm}

\begin{proof}
    Suppose for the sake of contradiction that there is $\alpha \in \QQ$ such that $\alpha^2 = 2$. We see that $\alpha = \frac{m}{n}$ for some $m, n \in \ZZ$ such that $\gcd(m, n) = 1$.

    Since $a^2 = 2$, we have $\frac{m^2}{n^2} = 2$, implying that $m^2 = 2n^2$, or $2 | m^2$. As $2$ is prime, we see that $2 | m \implies m = 2p$ for some integer $p$. Then, $(2p)^2 = 2n^2 \implies 4p^2 = 2n^2 \implies n^2 = 2p^2$. Once again, we have that $2 | n$ from above, so $m, n$ are both even. This contradicts the claim that $m, n$ are coprime, so $\sqrt 2$ cannot be expressed in a rational form. 
\end{proof}

This motivates the concept of incompleteness.

\begin{defn}[Incompleteness]
    Let $S$ be an ordered set, and let $A \subseteq S$. 
    \begin{enumerate}
        \item An element $\beta \in S$ is an \vocab{upper bound} for $A$ if $\alpha \leq \beta, \forall \alpha \in A$. 

        Then, we say that $A$ is \vocab{bounded above}.
        \item An element $\beta \in S$ is a \vocab{lower bound} for $A$ if $\alpha \geq \beta, \forall A \in A$. 
        
        Then, we say that $A$ is \vocab{bounded below}.
    \end{enumerate}
\end{defn}

\begin{defn}[Supremum]
    Suppose $S$ is an ordered set and $A \in S$ is bounded above. Suppose $\exists B \in S$ such that:
    \begin{enumerate}
        \item $\beta$ is an upper bound for $A$.
        \item If $r$ is another upper bound, then $r \geq \beta$.
    \end{enumerate}
    Then, we will call $\beta$ the least upper bound, or the \vocab{supremum} ($\sup$) of $A$.
\end{defn}

The greatest lower bound is then called the \vocab{infimum} ($\inf$) of $A$.

\begin{remark}
    Supremum and infimum may not exist or belong to $A$.
\end{remark}

\begin{defn}[Completeness]
    An ordered set $S$ is said to have the least upper bound property, or \vocab{completeness}, if every upper bounded set has a supremum in $S$.
\end{defn}

\subsection{Real Numbers}
\begin{thm}[Real Numbers]
    There is a unique ordered field $(\RR, +, \cdot, \leq)$ that has the following properties:
    \begin{enumerate}
        \item Completeness.
        \item $\QQ \subseteq \RR$ is an ordered subfield; i.e., $(+, \cdot, \leq)$ restricted to $\QQ$ are the usual $(+, \cdot, \leq)$ on $\QQ$.
    \end{enumerate}
\end{thm}

Lecturer says we will be using the result, and not the proof.

There's another theorem with properties of real number arithmetic but honestly I'm too lazy to write it as of now so you'll see it later.

\subsection{Consequences of the Completeness Axiom}

\subsubsection{Existence of Infimum}
\begin{thm}
    Let $E \subset \RR$ be a set bounded below. Then $\inf E$ exists in $\RR$.
\end{thm}
    
\begin{proof}
    Define $-E$ to be $\{-x | x \in E\}$. Then, we see that $-E$ must be bounded above, implying that it must have a supremum by definition of completeness. Then, we let $\sup -E = \beta$. 

    Then, for any $\alpha$ such that $x \geq \alpha \forall x \in E$, then we have $y \leq -\alpha \forall y \in E$. We see that $-\alpha \geq \beta$ by definition of supremum, so $\alpha \leq -\beta$. As a result, $\inf E = -\beta$.
\end{proof}

\newpage
\section{Lecture 3}
\subsection{Archimedean Property}
\begin{thm}[Archimedean Property]
    If $a > 0$ and $b > 0$, then for some positive integer $n$, we have $na > b$.
\end{thm}

\begin{proof}
    Assume that the Archimedean Property fails. Then, for all positive integers $n$, we have $na < b$ for some positive $a$ and $b$. 

    Now, let's observe $S = \{na | n \in \NN\}$. Then, let $b = \sup S$, which must exist by completeness.

    Consider $b - a$. Since $b$ is a supremum and $a$ is positive, then $\exists s \in S$ such that $s > b - a$. However, $a + s$ must also be in $S$ by definition, and $a + s > a + (b - a) = b$, contradicting the claim that $b = \sup S$.
\end{proof}

\begin{corollary}
    \begin{enumerate}
        \item If $a > 0$, then $\exists n \in \NN$ such that $\frac{1}{n} < a$.
        \item If $b > 0$, then $\exists n \in \NN$ such that $b < n$.
    \end{enumerate}
\end{corollary}

\begin{thm}[Density of $\QQ$]
    If $a, b \in \RR$ and $a < b$, then $\exists r \in \QQ$ such that $a < r < b$.
\end{thm}

\begin{proof}
    This is an exercise for the reader. Till I fill the proof in.
\end{proof}

\begin{thm}[Existence of $n$th roots]
    Given any $\alpha \in \RR$, $\alpha > 0$, and any $n \in \NN$, there's a $\beta \in \RR$ s.t. $\beta^n = \alpha$.
\end{thm}

\begin{proof}
    This is an exercise for the reader until I feel like writing more about this.
\end{proof}

\begin{corollary}
    \begin{enumerate}
        \item $b_1, b_2 > 0$ s.t. $b_1^n = b_2^n$. Then, $b_1 = b_2$.
        \item If $a, b > 0$, then $\sqrt[n]{ab} = \sqrt[n]{a}  \sqrt[n]{b}$.
    \end{enumerate}
\end{corollary}

\subsection{(Gates to) Infinity}
We define the set of \vocab{extended reals} to be $\RR^* = \RR \cup \{-\infty\} \cup \{\infty\}$, with the extended order $-\infty < \alpha < \infty$, $\forall \alpha \in \RR$.

Then, $\infty$ is an upper bound for any $E \subset \RR$ and $-\infty$ is a lower bound for any $E \subset \RR$. 

We can extend the definition of $\sup$ and $\inf$ such that
\begin{itemize}
    \item $\sup E = \inf$ if $E$ is not bounded above, and 
    \item $\inf E = -\inf$ if $E$ is not bounded below.
\end{itemize}

Note that $\RR^*$ does not form a field. As a result, we cannot apply a theorem or exercise stated for real numbers to $\infty$, $-\infty$. This set doesn't have an algebraic structure.

We also denote unbounded intervals using $-\infty, \infty$ instead of real numbers.

\begin{remark}
    Let $S$ be any nonempty subset of $\RR$. The symbols $\sup S$ and $\inf S$ always make sense. If $S$ is bounded above, then $\sup S$ is a real; otherwise, it is $+\infty$. Same logic applies to lower bounds and $-\infty$.

    Moreover, the statement $\inf S \leq \sup S$ also always makes sense.
\end{remark}
\end{document}