\documentclass{article}
\usepackage[utf8]{inputenc}
\usepackage{fancyhdr}
\usepackage{graphicx}
\usepackage{dirtytalk}
\usepackage[portrait, margin=0.5in, top=0.8in, bottom=0.3in]{geometry}
\usepackage{amsmath, amssymb}
\setlength{\parskip}{1em}
\setlength{\parindent}{0cm} 
\usepackage[mdthm,simplethm]{test}
\usepackage[utf8]{inputenc}
\usepackage{float}
\usepackage{physics}
\usepackage{sectsty}
\usepackage{tikz}
\usepackage{circuitikz}
\allsectionsfont{\normalfont\sffamily\bfseries}

\setlength{\parindent}{0pt}
\pagestyle{fancy}
\lhead{Problemset 3}
\rhead{Albert Ye}

\title{Problemset 3}
\author{albert ye}
\date{May 2022}

\begin{document}
	\maketitle
	\section{Build-Up Error?}
	A counterexample to this would be a graph with $4$ vertices and $2$ edges 
	such that $1, 2$ and $3, 4$ both have one edge between them. All vertices 
	have a degree of at least one, but there is no connection between the 
	two components so the graph is disconnected.

	This proof is incorrect because it relies on a specific construction for 
	a graph of $n+1$ vertices. It relies on there being $n+1$ vertices such that 
	the last vertex is linked to another vertex, and the other $n$ nodes being connected.
	However, the logic of the proof does not apply for all constructions, and there are 
	cases where the induction would fail. For example, consider the case where instead of
	going to another one of the $n$ nodes, the $n+1$th vertex had an edge 
	connecting to itself. Then, we would not be able to assume that the inductive step holds. 

	Generally in an induction, the inductive step must cover all possible cases for 
	the $k+1$ case, as opposed to just a specific construction that works.

	\section{Tournament}
	For a (really boring) two-player tournament, we would have $2$ vertices connected by 
	one edge, Regardless of who wins, we can traverse the graph by going from the losing vertex 
	to the winning vertex.

	Assume that the case holds for $n$ players. Then, we can add the $n+1$th player as a vertex 
	that has edges leading to or from all $n$ existing players. Assume WLOG that the Hamiltonian path 
	is $1 \to 2 \to \ldots \to n$, implying that player $1$ loses to player $2$ and so on to player $n$.

	Clearly, if player $n+1$ beats player $n$, then we have another Hamiltonian path. If $n+1$ loses
	to $n$, then we can run the same comparison but with $n-1$. We will be back to the same two conditions:
	either $n+1$ beats $n-1$ and goes between $n-1$ and $n$ in the Hamiltonian path, or it loses and 
	gets handed down to $n-2$. We can repeat this all the way down to player 1. If player $n+1$ loses to 
	player $1$, it is placed at the beginning of the Hamiltonian path. Regardless of how player $n+1$ plays,
	it will always have a place in our Hamiltonian path. Hence, we are done by induction. \qed
	\section{Proofs in Graphs}
	\begin{enumerate}[label=\alph*.]
		\item For the base case $n=2$, we have one vertex leading to another vertex. The 
		destination vertex will be reachable from the other vertex in distance 1.

		Assuming that we have $n$ vertices such that all vertices reach vertex $k$ in 
		at most two moves. Equivalently, all vertices are a distance $2$ from $k$.
		Then, for vertex $n+1$, we have a few possible constructions:

		The first would be if $n+1$ led to $k$. Then, we would be done because $n+1$ 
		would lead to $k$ in just one move. 

		The second would be if $k$ led to $n+1$, but $n+1$ led to a vertex $j$ that led to $k$.
		The claim would still hold because $n+1$ is a distance of $2$ away from $k$.
		
		Finally, if $n+1$ does not lead to any vertices a distance of $0$ or $1$ from $k$, that 
		means it only leads to vertices a distance $2$ from $k$. Therefore, $k$ and all vertices 
		leading to $k$ must have a direct connection to $n+1$. All vertices of distance $2$ from $k$
		must directly connect to a vertex of distance $1$ from $k$ by definition, so they must 
		also be a distance of $2$ from $n+1$.
		
		All possible constructions of $n+1$ nodes lead to some vertex being a distance of at most $2$ 
		from every other vertex, so the claim is proven by induction. \qed
		\item Denote the odd-degree vertices as $o_1, o_2, \ldots, o_{2m}$. Then, we add $m$ 
		"phantom" edges $(o_1, o_2), (o_3, o_4), \ldots, (o_{2m-1}, o_{2m})$, making the degree
		of $o_i$ even for all $i \in [1, 2m]$. 

		Now, we will have an Eulerian tour containing all of the real and phantom edges. Without
		loss of generality, assume that this Euler tour ends with a phantom edge. Then, when 
		we remove the $m$ phantom edges from the graph we have $m$ walks that all collectively
		traverse the whole graph.
		\item If $G$ is bipartite, then assume it has two sides $V_1$ and $V_2$. A move from 
		a node in $V_1$ would necessarily end up in $V_2$ by definition, and a move from $V_2$ 
		would necessarily end up in $V_1$, so an even number of moves is needed to go between 
		two nodes in $V_1$. Similarly, an even number of moves is needed to traverse a tour in $G$.

		Now, assume there are no tours of odd length in $G$. Fix an arbitrary vertex $v$ in a 
		connected component $C$. Then, let's color the nodes in $C$ that are an odd distance from $C$ 
		black, and color the nodes in $C$ that are an even distance from $C$ white. Because 
		the cycle length is even, we see that black nodes necessarily link only with white nodes, 
		and vice versa. Therefore, the graph that results by iterating over all vertices $v$ and 
		components $C$ must be bipartite. \qed
	\end{enumerate}
	\section{Planarity and Graph Complements}
	\begin{enumerate}[label=\alph*.]
		\item $\overline G$ must have all $\frac{v(v-1)}{2}$ edges of $K_v$ except 
		for the $e$ edges in $G$, so there are a total of $\boxed{\frac{v^2-v-2e}{2}}$ edges.
		\item Using Euler's theorem, we have $e \leq 3v - 6$ for a planar graph with $e$ edges 
		and $v$ vertices. For a graph with 13 vertices, 
		a planar graph can have at most $3 \cdot 13 - 6 = 33$ edges. However, there are 
		a total of $\frac{13 \cdot 12}{2} = 78$ edges. Regardless of how many edges we 
		give $G$, we cannot keep $G$ as a planar graph while also having a planar graph $\overline G$
		because $\overline G$ would need at least $78-33 = 45$ edges.

		For $13 + k$ edges, we can have $p = 2(33 + 3k)$ total edges to keep $G, \overline G$ planar and need
		$q = \frac{k^2-k}{2}$ edges total between $G$ and $\overline G$. As $p$ grows 
		linearly with $k$ and $q$ grows quadratically with $k$, we can ensure that $q > p$ must hold.
		Therefore, we cannot have both $G, \overline G$ be planar for $v > 13$.
		\item Imagine $\overline G = K_5$. We know that $\overline G$ cannot be planar 
		by Kuratowski's theorem. Then, $G$ would have $p \geq 8$ remaining vertices all completely connected. 
		This means that we can pick $5$ vertices in $G$ such that they are all interconnected
		with each other, implying that $G$ also has a subgraph of $K_5$. This construction 
		leads to a case where neither $G$ nor $\overline G$ are planar, contradicting the claim. \qed
	\end{enumerate}
	\section{Touring Hypercube}	
	\begin{enumerate}[label=\alph*.]
		\item For a hypercube in $n$ dimensions, each vertex connects to exactly $n$ 
		other vertices, one vertex per bit. Therefore, if $n$ is even, all vertices have 
		even degree so an Eulerian tour is possible. Otherwise, no vertices will have even 
		degree so an Eulerian tour is not possible. Therefore, there exists an Eulerian 
		tour in an $n$-D hypercube if and only if $n$ is even. \qed
		\item Let's once again define each vertex of the cube by a number from $00\ldots0_2 = 0$
		to $11\ldots1_2 = 2^n-1$ with $n$ zeroes and $n$ ones, respectively. Now, for a $0$-dimensional
		hypercube we just have a single point (which automatically has a Hamiltonian path).

		We can use induction to prove a Hamiltonian path exists for $n+1$ dimensions given the 
		Hamiltonian path for $n$ dimensions.
		WLOG, assume that the Hamiltonian path starts at vertex $0$ and ends at vertex $2^n-1$ for the $n$-
		dimensional hypercube.

		Now, for each vertex of the $n+1$-dimensional hypercube, consider the last $n$ vertices. 
		We find that we have two copies of each number from $00\ldots0$ to $11\ldots1$ such that 
		one copy is a vertex $v < 2^n$ (or first digit $0$), and one copy is a vertex $v \geq 2^n$
		(first digit 1). We must then have two copies of the Hamiltonian path from vertex $0$ to vertex $2^n-1$, 
		one for the vertices with first digit $0$ and one for those with first digit $1$.
		
		We can use this to construct a Hamiltonian path for an $n+1$-dimensional hypercube:
		start at vertex $0$ and use the Hamiltonian path for an $n$-dimensional hypercube to get to vertex 
		$2^n-1
		= 011\ldots1$. At this point, we can go from vertex $2^n-1 = 011\ldots1$ to vertex $2^{n+1}-1 = 
		111\ldots1$.
		However, we also know there exists a Hamiltonian path from vertex $2^n+0 = 2^n$ to vertex 
		$2^n+2^n-1 = 2^{n+1}-1$. We can reverse every edge along this path to get a Hamiltonian 
		path from $2^{n+1}-1$ to $2^n$. Since all vertices are covered by this traversal, we have a
		Hamiltonian path for an $n+1$-dimensional hypercube given an $n$-dimensional one. \qed
	\end{enumerate}

	\section{Connectivity}
	\begin{enumerate}[label=\alph*.]
		\item We prove the stricter claim that for any two non-adjacent vertices $u, v$, there must 
		be a third vertex $w$ such that $u,v$ are both adjacent to $w$. Let the set of adjacent nodes 
		to $u$ be $U_a$, and the set of adjacent nodes to $v$ be $V_a$. Then, $|U_a| + |V_a| \geq n-1
		\implies |U_a| + |V_a| > n-2 = |U_a \cup V_a|$, so $|U_a \cap V_A| \geq 1$ must hold. Thus,
		if all non-adjacent vertices $u, v$ satisfy $\deg u + \deg v \geq n-1$, then there must be 
		some vertex $w$ that is adjacent to both $u$ and $v$. 

		If $u$ can reach $v$ for all non-adjacent pairs of vertices $(u, v)$, then the graph must 
		be connected by definition. \qed
		\item In the following graph with $4$ vertices, all vertices have degree $1$ so all 
		pairs of vertices have a total degree of $4-2 = 2$. However, this graph is evidently not connected:
		\begin{center}
			\begin{tikzpicture}
				\clip (-1, -1) rectangle (8, 2.1);
				\node[circ] (ma) at (2, 0) {};
		        \node[circ] (mb) at (2, 2) {};
		        \node[circ] (mc) at (4, 2) {};
		        \node[circ] (md) at (4, 0) {};
				\draw (ma) -- (mb);
				\draw (mc) -- (md);
			\end{tikzpicture}
		\end{center}
		\item If all vertices have degree at least $\frac{n}{2}$, then for every non-adjacent 
		pair of vertices $u, v$, we would have $\deg u + \deg v \geq \frac{n}{2} + \frac{n}{2} > n-1$.
		Therefore, by part (a) the claim must hold.
		\item By the Handshake Lemma, the number of vertices with odd degree.
		Thus, it is impossible for the sum of degrees in a connected component 
		to be odd. Therefore, if there are only 2 nodes with odd degree in a graph, then both will 
		need to be in the 
		same connected component to ensure that the sum of degrees in said component remains even.
	\end{enumerate}

	\section{Edge Colorings}
	\begin{enumerate}[label=\alph*.]
		\item One possible coloring is as follows:
		\begin{center}
			\begin{tikzpicture}
				\clip (-1, -1) rectangle (8, 3);
				\node[circ] (m1) at (2, 0) {};
	        	\node[circ] (m2) at (2, 2) {};
	        	\node[circ] (m3) at (4, 2) {};
	        	\node[circ] (m4) at (4, 0) {};
	        	\draw (m1) -- node[left]{1} (m2) -- node[above]{3} (m3) -- node[right]{1} (m4) -- node[below]{3} (m1);
	        	\draw (m2) -- node[above right]{2} (m4);
	        	\draw (m1) edge[out=-60, in=-30, looseness=2.5] node[below right]{2} (m3);
			\end{tikzpicture}
		\end{center}
		\item Given an edge $e$, no other edges adjacent to $e$ can have the same color. Each end of the edge 
		$e$ can have at most $d-1$ other edges in order to have maximum degree of $d$. Therefore,
		the maximum number of edges adjacent to any current edge is $2(d-1) = 2d-2$. If $2d-1$ colors 
		were used, then there will always exist an open color for every edge.
		\item In a tree, each edge $e$ between vertices $p, v$ (where $p$ is closer to the root than $v$)
		can still have $d-1$ adjacent edges on each end, but the nature of 
		the connected vertices differ on each end. Assume that both sides of $e$ have $d-1$ adjacent edges. 
		Then the $d-1$ adjacent nodes connected to $v$ are child nodes and the $d-1$ adjacent nodes 
		connected to $p$ are sibling nodes.

		By the definition of a tree, sibling nodes and child nodes cannot have edges between them, since 
		that would add a cycle to the graph. Therefore, the $d-1$ edges leading from $p$ and the $d-1$ 
		edges leading from $v$ do not affect each other, so we can make a stricter bound of $d$ colors
		for trees.
	\end{enumerate}

\end{document}