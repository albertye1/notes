\documentclass{article}
\usepackage[utf8]{inputenc}
\usepackage{fancyhdr}
\usepackage{graphicx}
\usepackage{dirtytalk}
\usepackage[portrait, margin=0.5in, top=0.8in, bottom=0.3in]{geometry}
\usepackage{amsmath, amssymb}
\setlength{\parskip}{1em}
\setlength{\parindent}{0cm} 
\usepackage[mdthm,simplethm]{test}
\usepackage[utf8]{inputenc}
\usepackage{float}
\usepackage{physics}
\usepackage{sectsty}
\usepackage{tikz}
\usepackage{circuitikz}
\allsectionsfont{\normalfont\sffamily\bfseries}

\setlength{\parindent}{0pt}
\pagestyle{fancy}
\lhead{Problemset 4}
\rhead{Albert Ye}

\title{Problemset 4}
\author{albert ye}
\date{May 2022}

\begin{document}
	\maketitle
	\section{Modular Practice}
	\begin{enumerate}[label=\alph*)]
		\item $9x+5 \equiv 7 \pmod{11} \implies 9x \equiv 2 \pmod{11} \implies x \equiv 2 \cdot 9^{-1} \pmod{11} 
			\implies x \equiv 2 \cdot 5 \equiv \boxed{10} \pmod{11}$.
		\item $3x+15 \equiv 4 \pmod{21} \implies 3x \equiv -11 \pmod{21}$. However, as $\gcd(3, 21) = 3$, we will need
			$3x \equiv k \pmod{21}$	where $k \equiv 0 \pmod 3$. Therefore, there are no solutions to $3x \equiv -11 \pmod{21}$.
		\item \begin{align*}{}
			3x + 2y &\equiv 0 \pmod 7 \\
			2x + y &\equiv 4 \pmod 7 \\
			\implies 4x + 2y &\equiv 1 \pmod 7 \\
			\implies x &\equiv 1 \pmod 7. \\
			\text{Substituting for }x, \\
			2 + y &\equiv 4 \pmod 7 \\
			\implies y &\equiv 2 \pmod 7. \\
			\implies \boxed{x \equiv 1 \pmod 7} &, \boxed{y \equiv 2 \pmod 7}.
		\end{align*}
		\item $13^{2019} \equiv x \pmod{12} \implies 1^{2019} \equiv x \pmod{12} \implies \boxed{x \equiv 1 \pmod{12}}$.
		\item $7^{10} \equiv 1 \pmod{11}$ from Fermat's Little Theorem, so $7^{21} \equiv 7 \cdot 7^{20} \equiv \boxed{7 \pmod{11}}$.
	\end{enumerate}

	\newpage
	\section{Nontrivial Modular Solutions}
	\begin{enumerate}[label=\alph*)]
		\item Because $7^3 \equiv 0 \pmod 7$, we have $(7a+k)^3 \equiv (7a)^3 + (7a)^2k + (7a)k^2 + k^3 \equiv k^3 \pmod 7$. 
			Thus, if $j \equiv k \pmod 7$, then $j^3 \equiv k^3 \pmod 7$. It follows to check the first 7 perfect cubes modulo 7.
			Those would be $0^3 \equiv 0 \pmod 7, 1^3 \equiv 1 \pmod 7, 2^3 \equiv 1 \pmod 7, 3^3 \equiv 6 \pmod 7,
			4^3 \equiv 1 \pmod 7, 5^3 \equiv 6 \pmod 7,$ and $6^3 \equiv 6 \pmod 7$. Therefore, the only three possible 
			perfect cubes modulo 7 are $\boxed{0, 1, 6}$.
		\item $a^3 + 2b^3 \pmod 7$ can have a few possibilities. 

			Clearly, we can have $a \equiv b \equiv 0 \pmod 7$, which will give $a^3 + 2b^3 \equiv 0 \pmod 7$. Now,
			we claim that this is the only way to get $0 \pmod 7$. Clearly if $a^3 \equiv 0 \pmod 7$ and $b^3 \not\equiv 0 \pmod 7$
			then $a^3 + 2b^3 \not\equiv 0 \pmod 7$, and similarly for $b^3 \equiv 0 \pmod 7$ and $a^3 \not\equiv 0 \pmod 7$.
			The easiest way to prove this claim for cases where both $a^3 \not\equiv 0$ and $b^3 \not\equiv 0$ would be to loop over 
			all possibilities of $(a, b)$ and verify that none satisfy the condition. From part (a), we can group those 
			into cases where $a^3$ and $b^3$ equal either $1$ or $6$.

			\textbf{Case 1:} $a^3 \equiv 1 \pmod 7, b^3 \equiv 1 \pmod 7$. Then $a^3 + 2b^3 \equiv 1 + 2 \equiv 3 \pmod 7$.
			
			\textbf{Case 2:} $a^3 \equiv 1 \pmod 7, b^3 \equiv 6 \pmod 7$. Then $a^3 + 2b^3 \equiv 1 + 12 \equiv 13 \equiv -1 \pmod 7$.

			\textbf{Case 3:} $a^3 \equiv 6 \pmod 7, b^3 \equiv 1 \pmod 7$. Then $a^3 + 2b^3 \equiv 6 + 2 \equiv 8 \equiv 1 \pmod 7$.

			\textbf{Case 4:} $a^3 \equiv 6 \pmod 7, b^3 \equiv 6 \pmod 7$. Then $a^3 + 2b^3 \equiv 6 + 12 \equiv 18 \equiv 4 \pmod 7$.

		\item From part (b), we have that $a^3 + 2b^3$ only divides 7 when $a \equiv b \equiv 0 \pmod 7$. Therefore, we would need 
		$a \equiv b \equiv 0 pmod 7$ in order for $a^3 + 2b^3 = 7a^2b$ to hold. Let $a = 7p$ and $b = 7q$. Then, our equation 
		becomes $(7p)^3 + 2(7q)^3 = 7(7p)^2(7q) \implies 343(p^3+2q^3) = 2401p^2q \implies p^3 + 2q^3 = 7p^2q$. This means that $7$ 
		must also divide $(p, q)$. 
		
		Therefore, in order for $(a, b)$ to satisfy this relation, we will need $7^k | a, b$ for all $k \in \ZZ^*$. The only values 
		of $a$ and $b$ that can satisfy this condition would be $a = 0, b = 0$, so there are no nontrivial solutions for $(a, b)$.
		\qed
	\end{enumerate}

	\newpage
	\section{Wilson's Theorem}
	For the if direction, we know that $(p-1)! = 1 \cdot 2 \cdot \cdots \cdot (p-1)$. Now, we examine all of these indices modulo $p$. Because 
	$p$ is prime, we know that every residue from $1$ to $p-1$ has an inverse modulo $p$. However, we also know that $1$'s 
	inverse is $1$, and $p-1$'s inverse is $p-1$ because $(p-1)(p-1) = p^2-2p+1 \equiv 1 \pmod p$. Therefore, when multiplying 
	from $1$ to $p-1$ modulo $p$, all of the inverse pairs cancel out except for $(p-1, p-1)$ since there is only one instance
	of $p-1$. Therefore, if $p$ is prime then $(p-1)! \equiv p-1 \equiv -1 \pmod p$.

	For the only-if direction, assume for the sake of contradiction that $p$ is composite. We fix a prime factor $q$ of $p$ and notice that
	 $(p-1)! \equiv 0 \pmod q$
	since $q$ must be a factor of $(p-1)!$. Therefore, $(p-1)!$ cannot have a residue of $p-1 \pmod q$ since $q \not\vert p-1$, contradicting
	our initial claim. \qed

	\newpage
	\section{Fermat's Little Theorem}
	From Fermat's Little Theorem, we have that $n^7-n \equiv 0 \pmod 7$, so all that is left to prove is that $n^7-n \equiv 0 \pmod 6$.
	
	Divide $6$ into $2$ and $3$. Then, from more applications of Fermat, we find that $n^2 \equiv n \pmod 2$ and $n^2 \equiv 1 \pmod 3$.
	Therefore, modulo $2$, we have $n^7 \equiv (n^2)(n^2)(n^2)(n) \equiv n^4 \equiv (n^2)(n^2) \equiv n^2 \equiv n \pmod 2$. Modulo $3$,
	we have $n^7 \equiv (n^2)^3(n) \equiv n \pmod 3$. Therefore, $n^7 - n \equiv 0 \pmod 2$ and $n^7 - n \equiv 0 \pmod 3$ so 
	$n^7 - n \equiv 0 \pmod 6$. \qed
	
	\newpage
	\section{Euler Totient Function}
	\begin{enumerate}[label=\alph*)]
		\item $\varphi(p) = p-1$ because all positive integers $k < p$ are relatively prime to $p$, and there are $p-1$ such integers.
		\item $\varphi(p^k) = p^k-1 - (p^{k-1} - 1) = p^k - p^{k-1}$. Take the $p^k-1$ numbers less than $p^k$ and remove the $p^{k-1}-1$ multiples of $p$.
		\item We observe that if there is a residue $k$ modulo $m$ such that $\gcd(k,m) = 1$, then $\gcd(k+am) = 1$ for all $a \in \ZZ$. Then, 
			let the set $M$ be the set of residues $k$ modulo $m$ such that $\gcd(k,m) = 1$, and the set $N$ be the set of residues $k$ modulo 
			$n$ such that $\gcd(k, n) = 1$. 

			Then, from the Chinese Remainder Theorem, since $\gcd(m, n) = 1$, then there is one $x$ modulo $mn$ that satisfies
			$x \equiv p \pmod m, x \equiv q \pmod n$. Therefore, for every $p \in M, q \in N$, there is one such $x$. Therefore, 
			$\varphi(mn) = |M| |N| = \varphi(m)\varphi(n)$. \qed
		\item Let's observe every prime factor separately. For $x = p_i^{\alpha_i}$, we have $\varphi(x) = p_i^{\alpha_i} - p_i^{\alpha_i-1}$
			from part (b). As we know from part (c) that $\varphi$ is multiplicative, we find that 
			\[\varphi(n) = \prod_{i=1}^{k} p_i^{\alpha_i} - p_i^{\alpha_i-1}.\]
			
			Simplifying gives us 
			\[\varphi(n) = \prod_{i=1}^{k} p_i^{\alpha_i-1}(p_i-1).\]
			\[\implies \varphi(n) = n\prod_{i=1}^{k} \frac{p_i-1}{p_i}.\] \qed
	\end{enumerate}

	\newpage
	\section{Euler's Totient Theorem}
	\begin{enumerate}[label=\alph*)]
		\item If $f(x) = ax \mod n$ is not a bijection, we have two values $m_i, m_j$ such that 
		$am_i \equiv am_j \pmod n$. Since $\gcd(a, n) \equiv 1$, we can multiply both sides 
		by $a^{-1}$ to get $am_ia^{-1} \equiv am_ja^{-1} \pmod n \implies m_i \equiv m_j \pmod n$.
		Therefore, all $am_i$ must map to a distinct value, or in other words, $f$ is a bijection. \qed
		\item From (a), we know that $\{am_1, am_2, \ldots, am_{\varphi(n)}\}$ is a permutation of 
		$\{m_1, m_2, \ldots, m_{\varphi(n)}\}$. Therefore, we have that
		\[\prod_{i=0}^{\varphi(n)}m_i \equiv \prod_{i=0}^{\varphi(n)}am_i \pmod{n}.\]
		The RHS evaluates to \[a^{\varphi(n)} \prod_{i=0}^{\varphi(n)}m_i \pmod{n},\] and as 
		$\prod_{i=0}^{\varphi(n)}m_i$ is coprime to $n$ by definition we can multiply by its inverse, giving us 
		$a^{\varphi(n)} \equiv 1 \pmod n$. \qed
	\end{enumerate}

	\newpage
	\section{Sparsity of Primes}
	Let's think of $x+i$ in terms of being divisible by the product of two primes. Therefore, for each 
	$x+i$, we would want $x+i \equiv 0 \pmod{p_{2i-1}p_{2i}}$ for all $i \in [1, k]$. Reducing to make the
	relations in terms of $x$, we have $x \equiv -i \pmod{p_{2i-1}p_{2i}}$. 
	We can come up with $2k$ distinct primes because there are infinite primes. 
	This means all $p_i$ are distinct, so the moduli are all coprime. 
	We have reduced the claim to $k$ modular equations with $k$ coprime moduli, so from the Chinese 
	Remainder Theorem there must exist an $x$ such that $x+1, x+2, \ldots, x+k$ are all not prime
	powers. \qed
	
	\newpage
	\section{Sundry}
	Consulted Yuchan Yang for help with problem 7.

\end{document}
