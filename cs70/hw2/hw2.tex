\documentclass{article}
\usepackage[utf8]{inputenc}
\usepackage{fancyhdr}
\usepackage{graphicx}
\usepackage{dirtytalk}
\usepackage[portrait, margin=0.5in, top=0.8in, bottom=0.3in]{geometry}
\usepackage{amsmath, amssymb}
\setlength{\parskip}{1em}
\setlength{\parindent}{0cm} 
\usepackage[mdthm,simplethm]{test}
\usepackage[utf8]{inputenc}
\usepackage{float}
\usepackage{physics}
\usepackage{sectsty}
\allsectionsfont{\normalfont\sffamily\bfseries}

\setlength{\parindent}{0pt}
\pagestyle{fancy}
\lhead{Problemset 2}
\rhead{Albert Ye}

\title{Problemset 2}
\author{albert ye}
\date{May 2022}


\begin{document}
\maketitle

\section{Airport}
For $n=1$, we must have a scalene triangle $\triangle ABC$. If $BC$ is 
the shortest edge, then $B$ will point to $C$ and $C$ will point to $B$. Then 
$A$ will point to the closer of $B$ and $C$, but nothing will point to $A$.

Given the claim holds for $n$, we seek to prove the claim for $n+1$. Let $A$
be an airport such that there is no flight towards it. If we were to add two 
vertices such that none of them affected $A$, we would still have no flights 
destined to $A$. Thus, the claim still holds for such cases. 

Now, let's add one vertex $B$ close to $A$ such that a flight from $B$ would be destined to $A$. Then, 
$B$ would be the closest vertex to $A$ as well, so we would have a cycle of size $2$
and a remaining network of size $2n$. Now, let's add the final vertex $C$. 

If $C$ is
closer to a vertex on the cycle of size $2$, we would be able to split the problem 
into a network of size $3$ and a network of size $2n$, and the part of size $3$ would 
necessarily have a vertex without inbound flights. Analogously, if $C$ is closer to 
the network of size $2n$, then we would split the graph into a cycle of size 2 
and a flight network of size $2n+1$, which we assume to have a vertex without inbound flights. 

Therefore, regardless of how we add two airports, there will 
always still be an airport without inbound flights for $n+1$ if there is such 
an airport for $n$. \qed

\section{Proving Inequality}
For $n \in \ZZ^+$, let $a_n = \frac{1}{3^1} + \frac{1}{3^2} + \cdots + \frac{1}{3^n}$.
We strengthen the inductive hypothesis, claiming that 
$a_n = \frac{3^0 + 3^1 + \cdots + 3^{n-1}}{3^n} < \frac{1}{2}$ for all $n$.
To see why it is less, consider base 3. $3^0 + 3^1 + \cdots + 3^{n-1} = 111\ldots1_3$ with $n$ ones.
Doubling that would give $222\ldots2_3$ with $n$ twos. This is strictly less than 
$100\ldots0_3$ with $n$ zeroes.

Plugging in $n=1$, we see that $a_1 = \frac{3^0}{3^1} = \frac{1}{3}$ is true. Now,
we can use induction to prove this over all positive integer $n$. Given 
$a_n = \frac{3^0 + 3^1 + \cdots + 3^{n-1}}{3^n}$, we add $\frac{1}{3^{n+1}}$ to get 
$a_{n+1}$. Thus, we have 
\begin{align*}
	a_{n+1} &= \frac{3^0 + 3^1 + \cdots + 3^{n-1}}{3^n} + \frac{1}{3^{n+1}} \\
	&= \frac{3^1 + 3^2 + \cdots + 3^n}{3^{n+1}} + \frac{1}{3^{n+1}} \\
	&= \frac{1 + 3^1 + 3^2 + \cdots + 3^n}{3^{n+1}} \\
	&= \frac{3^0 + 3^1 + \cdots + 3^n}{3^{n+1}}.
\end{align*}

We see that the formula still holds for $n+1$ given $n$, so $a_n < \frac{1}{2}$ must be 
true for all $n \in \ZZ^+$. \qed

\section{Inductive Charging}
\begin{enumerate}[label=\alph*.]
	\item Assume for the sake of contradiction that $f_i < d_i$ for all $i$. 
		Then, $\sum_{i=0}^{n} f_i < \sum_{i=0}^n d_i = D$, meaning that 
		there is not enough total fuel for one car to circle the track. 
		Since that is a given for this problem, we have a contradiction. 
		Thus, at least one car must have enough fuel to reach the next. \qed
	\item We once again use induction on $n$. Start with $n=1$, which clearly works 
		as $n=1$ means $f_1 = D$ in order for the one car to have enough fuel to go 
		around the track. 

		Now, given that $n$ satisfies the claim, we try to prove the same for $n+1$. 
		From part (a), we know that one car has enough fuel to travel to the next car. 
		Without loss of generality, let that car be the first one. Then $f_1 > d_1$. 

		Our first car can thus reach the second car, and then transfer the fuel from the second 
		car. Thus, we need a total fuel count of $f_1 + f_2$ to go a distance of 
		$d_1 + d_2$. We can combine the two into one segment: $(f_1+f_2) + f_3 + \cdots + f_{n+1} = (d_1+d_2) 
		+ \cdots + d_{n+1}$. Doing this essentially reduces the problem from $n+1$ cars into $n$ cars,
		which we can assume already satisfies this claim. \qed 
\end{enumerate}

\section{A Coin Game}
Clearly, if $n = 1$, then the resulting score must be 0 since there are no ways 
to split the coins. We seek to prove that if $n = a+b$, then $\frac{a(a-1)}{2} + \frac{b(b-1)}{2} + ab =
\frac{n(n-1)}{2}$. 

Adding this together, we have
\begin{align*}
	\frac{a(a-1)}{2} + \frac{b(b-1)}{2} + ab &= \frac{a^2-a + b^2-b}{2} + ab \\
	&= \frac{a^2 + b^2 + 2ab - a - b}{2} \\
	&= \frac{(a+b)^2 - (a+b)}{2} \\
	&= \frac{n^2 - n}{2} \\
	&= \frac{n(n-1)}{2}.
\end{align*}

Using this, we can divide a stack of $n$ coins into stacks of smaller coins. Since 
we already know $n=1$ must hold true, our claim that the score will always be 
$\frac{n(n-1)}{2}$ holds for all $n$ from induction. \qed

\section{Pairing Up}

Let's construct the solution for just one pair. 

Imagine the case where $J_1$ prefers $C_2 > C_1$ and $J_2$ prefers 
$J_2 > J_1$, while $C_1$ prefers $J_1 > J_2$ and $C_2$ prefers $J_2 > J_1$. 
Then the pairs $(J_1, C_1), (J_2, C_2)$ are a stable matching since the candidates 
both prefer the jobs they've received. But the pairs $(J_1, C_2), (J_2, C_1)$ are 
also a stable matching because the jobs both prefer the candidates they've received. 
Therefore, we can construct an instance of $2$ jobs and $2$ candidates 
with $2^{2/2} = 2$ stable matchings.

For a larger $n$ jobs and $n$ candidates, we can divide them into $\frac{n}{2}$ pairs $(i, i+1)$ 
and run the same preference scheme between $J_i, J_{i+1}$ and $C_i, C_{i+1}$. To avoid 
some other job $J_k$ or candidate $C_k$ causing the matchings between $J_i, J_{i+1}$ and 
$C_i, C_{i+1}$ to be rogue, we can put $J_i$ and $J_{i+1}$ as the top preferences 
for $C_i$ and $C_{i+1}$, and vice versa.

Assume there are $2^{n/2}$ stable matchings for $n$ jobs and $n$ candidates.
When we add a pair $(n+1, n+2)$, we get two possible stable matchings: $(J_{n+1}, C_{n+1}),
(J_{n+2}, C_{n+2})$ and $(J_{n+1}, C_{n+2}), (J_{n+2}, C_{n+1})$. For each of the
$2^{n/2}$ original stable matchings, we can choose one of the possible stable matchings 
between $n+1$ and $n+2$ without affecting any of the other pairs. This gives us 
$2^{(n+2)/2}$ stable matchings for $n+2$ jobs and candidates.


\section{Nothing Can Be Better Than Something}
\begin{enumerate}[label=\alph*.]
	\item Let the jobs be $J_1, J_2, \ldots, J_n$ and let the matching  
		candidates by $C_1, C_2, \ldots, C_n$. Let's introduce a "phantom" 
		set of jobs $J_1', J_2', \ldots, J_n'$ where $J_i'$ is the job 
		$C_i$ takes if it does not like any of the jobs it gets. We 
		introduce another phantom set of candidates $C_1', C_2', \ldots, C_n'$ 
		where $C_i'$ is the candidate job $J_i$ takes if it does not like 
		any of its candidates. 

		Assume there is a stable matching where $k$ candidates $C_i$ that do not have a preferred job. 
		Then, obviously, these candidates would 
		match to $J_i'$. But when these candidates unmatch from their jobs, there 
		will be $k$ jobs that will have to match to phantom candidates $C_i'$. 
		We now show that this arrangement meets all five criteria for a stable matching.

		In this case, there will be no matched entity who prefers being unmatched over 
		being with their current partner. All candidates are the top preference of their 
		matching phantom jobs (and vice versa), so by the improvement lemma no candidate 
		or job will ever be matched with a candidate they do not want to be matched with.

		Assume there'll be a matched job $J$ and unmatched candidate $C$ that prefer each 
		other. Then, $J$ would have to be higher on the candidate's list 
		than the phantom job $J'$ that candidate did choose. If $J$ prefers $C$ over a candidate 
		it matched with, then $C$ would have earlier gotten an offer from $J$, 
		contradicting the claim that $C$ is unmatched. This logic applies similarly for a
		matched candidate and unmatched job.

		Finally, assume there is an unmatched job $J$ and unmatched candidate $C$ that prefer to 
		be with each other over their phantom matches $C'$ and $J'$ respectively. 
		Then $J > J'$ for $C$ and $C > C'$ for $J$. Then, $J$ would have at some point 
		extended an offer to $C$ before settling for $C'$, and $C$ would have accepted it 
		over $J'$ because $J$ is preferred. This contradicts the claim that $J, C$ are unmatched.
		
		Then, there are $n-k$ "phantom" 
		jobs and candidates that still 
		need to be matched, but we can 
		necessarily create a stable matching out 
		of all of them since all pairs of $x$ jobs and $x$ candidates can be stably matched. \qed 
	\item Let's assume for the sake of contradiction that we have a candidate
		$C$ that gets no match and has to go for $J'$ in one stable matching,
		but $C$ does get a match $J_1$ in another stable matching. 
	
		This means that the original match to $J_1$, which we can call $C_1$, has to be matched 
		to another job $J_2$. 
		Whenever we rematch a candidate $C_i$ to another job $J_{i+1}$, we will have to displace 
		a candidate $C_{i+1}$ to another job $J_{i+2}$. But as the set 
		of candidates and jobs are finite, there will be a candidate $C_n$ that cannot be matched to
		a better job, creating a rogue couple between $C_n$ and $J_n$. \qed 
\end{enumerate}

\section{Sundry}
I worked on this alone, but I discussed the approaches towards some problems with my roommate
Saathvik Selvan.
\end{document}