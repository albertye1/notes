\documentclass{scrartcl}
\usepackage[utf8]{inputenc}
\usepackage{fancyhdr}
\usepackage{graphicx}
\usepackage{dirtytalk}
\usepackage[portrait, margin=1in]{geometry}
\usepackage{amsmath, amssymb}
\setlength{\parskip}{1em}
\setlength{\parindent}{0cm} 
\usepackage[mdthm,simplethm]{test}
\usepackage[utf8]{inputenc}
\usepackage{float}
\usepackage{physics}
\setlength{\parindent}{0pt}

\title{Scheme}
\author{albert ye}
\date{May 2022}

\begin{document}
\maketitle
Scheme is a dialect of lisp from the '70s but somehow
Berkeley decides it's still relevant
to teach.
\section{Scheme expressions}
Scheme has two types of expressions, which can
either be \vocab{primitive} or \vocab{combinations}.

From what I gather \vocab{primitive expressions} are just
one variable/function (such as "2" or "quotient") whereas
combinations are adding variables and functions to evaluate
things. Call expressions include an operator and 0 or more
operands in parentheses.

For example: \texttt{(quotient 10 2)} is a valid expression
that returns \texttt{5}.
\section{Special forms}
A \vocab{special form} is a combination that isn't a call 
expression.
\begin{itemize}
	\item if expressions: \\ \texttt{(if <predicate> <consequent> <alternative>)} 
	\item and/or: \\ \texttt{(and <e1> ... <en>)} \\ \texttt{(or <e1> ... <en>)}
	\item binding symbols: \\ \texttt{(define <symbol> <expression>)}
	\item new procedures: \\ \texttt{(define (<symbol> <formal parameters>) <body>)}
\end{itemize}
\subsection{define form}
Takes an expression and binds the value of the expression to a name (which must be 
a valid Scheme symbol)

For example, \texttt{(define x 2)} sets \texttt{x}
to the value \texttt{2}.

Using the \texttt{let} command allows you to set the value 
temporarily without binding it to the name. For example:
\texttt{(define c (let ((a 3) (b (+ 2 2))) (sqrt (+ (* a a) (* b b)))))}

\subsection{define procedure}
Constructs a new procedure with specified parameters
and uses the \texttt{body} expressions as its body
which it binds to \texttt{name}.

For example \texttt{(define (double x) (* 2 x))} defines 
a function 

A \vocab{lambda expression} is just an anonymous procedure.

\subsection{list procedure}
\say{you use a lot of recursion \\
and lists are linked lists} -chez

The \vocab{list} procedure takes in an arbitrary number
of arguments and constructs a list with the values of 
these arguments.

\texttt{(list 1 2 3)} makes a list \texttt{1, 2, 3}

\texttt{(list 1 (list 2 3) 4)} makes a list \texttt{1, (2, 3), 4}

\texttt{(list (cons 1 (cons 2 nil)) 3 4)} makes \texttt{(1, 2), 3, 4}

\subsection{car and cdr and cons}
Not related to quotation but it first showed up here; 
\vocab{car} is the first element of a list, \vocab{cdr}
is the list of remaining elements after the first, and 
\vocab{cons} is short for construct which appends to a list. 

\vocab{car} and \vocab{cdr} pretty much make all lists follow the
linked list paradigm.

If you only want to construct one element 
with another element, then you can just use \texttt{(cons 1 nil)}.

\section{Quotation}
Symbols typically refer to values, and quotation is used to refer 
to the symbols directly. 

\texttt{(list 'a 'b)} makes the list \texttt{a, b}
while \texttt{(list a b)} makes the list \texttt{1, 2}
if $a=1, b=2$.

\texttt{'(neverseen)} returns "neverseen" because it's still a symbol,
the only caveat is that nothing can be done with "neverseen".

\section{List Processing}
\begin{itemize}
	\item \texttt{append a b}, adds the value \texttt{b} to the list
		\texttt{a}.
	\item \texttt{filter cond list}, returns a list of all $x \in list$
		that satisfy \texttt{cond}
	\item \texttt{apply func list} applies \texttt{func} over all
		values in \texttt{list}.
	\item \texttt{map lambda(x) list} returns the elements of a list
		after applying a lambda over it.
	\item \texttt{eval exp} basically allows you to run 
		a program \texttt{exp} you just put into a list
\end{itemize}

\end{document}