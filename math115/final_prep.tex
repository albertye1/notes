\documentclass{article}
\usepackage[utf8]{inputenc}
\usepackage{fancyhdr}
\usepackage{graphicx}
\usepackage[portrait, margin=0.5in, top=0.8in, bottom=0.3in]{geometry}
\usepackage{amsmath, amssymb}
\setlength{\parskip}{1em}
\setlength{\parindent}{0cm} 
\usepackage[mdthm,simplethm]{test}
\usepackage[utf8]{inputenc}
\usepackage{float}
\usepackage{physics}
\usepackage{sectsty}
\usepackage{tikz}
\usepackage{circuitikz}
\allsectionsfont{\normalfont\sffamily\bfseries}

\usepackage{hyperref}
\usepackage{nameref}
\hypersetup{
    colorlinks=true,
    linkcolor=blue,
    filecolor=magenta,      
    urlcolor=cyan,
    pdftitle={Math 115 Notes},
    pdfpagemode=FullScreen,
    }

\urlstyle{same}

\setlength{\parindent}{0pt}
\pagestyle{fancy}
\lhead{Math 115 - Final Review}
\rhead{Albert Ye}

\title{Math 115 - Final Review}
\author{albert ye}
\date{\today}

\begin{document}
	\maketitle
	All of the material between the second midterm and the final.
	\section{Unit Groups}
	\section{Gauss Sums}
	\section{Jacobi Sums}
	Some initial definitions would help provide some context:
	\begin{definition}[Multiplicative Character]
		We call $\chi$ a \vocab{multiplicative character} if $\chi(ab) = \chi(a) \chi(b)$, over all $a, b \in F_p$.
	\end{definition}

	We have a couple other rules for this function:
	\begin{enumerate}
		\item $\chi(1) = 1$
		\item $\chi(a)$, over any $a$, is a $p - 1$th root of unity
		\item $\chi(a^{-1}) = \chi(a)^{-1} = \overline{\chi(a)}$.
	\end{enumerate}

	We also have the identity character $\varepsilon$, where $\varepsilon(a) = 1$ over all $a \in F_p$. 

	\begin{definition}[Jacobi Sum] 
		The \vocab{Jacobi sum} of two multiplicative characters $\chi, \lambda$, given $a, b \in F_p$ for some prime $p$, is $J(\chi, \lambda) = \sum_{a + b = 1} \chi(a) \lambda(b)$.
	\end{definition}

	There are also a couple theorems you should know, and one you should know the proof of:
	\begin{theorem}
		$\qquad$
		\begin{enumerate}
			\item $J(\epsilon, \epsilon) = p$.
			\item $J(\epsilon, \chi) = 0$.
			\item $J(\chi, \chi^{-1}) = - \chi(-1)$.
			\item If $\chi \lambda \neq \epsilon$, then \[J(\chi, \lambda) = \frac{g(\chi)g(\lambda)}{g(\chi)(\lambda)}.\] 
		\end{enumerate}
	\end{theorem}

	This one we need to prove: The first two statements can be done just by checking the sum over all $a, b \in F_p$ such that $a + b = 1$. 

	The third statement is true because $\chi(a) \chi^{-1}(b) = \chi(a) \chi(b^{-1}) = \chi\left(\frac{a}{b}\right)$, so we're summing over $\chi(a(1-a)^{-1})$. We see that this maps to every value but $-1$, so we have that our sum includes everything but $\chi(-1)$. But as the sum of everything is $0$, we have $J(\chi, \chi^{-1}) = -\chi(-1)$.

	The last statement is true because 
\end{document}
