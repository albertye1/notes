\documentclass{article}
\usepackage[utf8]{inputenc}
\usepackage{fancyhdr}
\usepackage{graphicx}
\usepackage{dirtytalk}
\usepackage[portrait, margin=0.5in, top=0.8in, bottom=0.3in]{geometry}
\usepackage{amsmath, amssymb}
\setlength{\parskip}{1em}
\setlength{\parindent}{0cm} 
\usepackage[mdthm,simplethm]{test}
\usepackage[utf8]{inputenc}
\usepackage{float}
\usepackage{physics}
\usepackage{sectsty}
\usepackage{tikz}
\usepackage{circuitikz}
\allsectionsfont{\normalfont\sffamily\bfseries}

\setlength{\parindent}{0pt}
\pagestyle{fancy}
\lhead{Math 115 - Introduction to Number Theory}
\rhead{Albert Ye}

\title{A Bad Introduction to Number Theory}
\author{albert ye}
\date{\today}

\begin{document}
\maketitle

\section{Lecture 1}
\begin{definition}
    An integer $p \neq 0, 1, -1$ is \vocab{prime} if the only integers which divide $p$ are $\pm 1$ and $\pm p$.
\end{definition}

Recall that the integers $\ZZ = \{\ldots, -3, -2, -1, 0, 1, 2, 3, \ldots\}$, $\NN = \{0, 1, 2, 3, \ldots\}$.

\begin{theorem}[Twin Prime Conjecture]
    There are infinitely many $p \in \NN$ such that $p$ is prime and $p + 2$ is prime.
\end{theorem}

Yitang Zhang proved bounded gaps between primes, so there are infinitely many prime $p, p + N$.

\begin{theorem}[Goldbach Conjecture]
    Every even number can be written as the sum of two primes.
\end{theorem}

Vinagradar proved that every odd number can be written as the sum of $3$ primes. The proof should use something called sieves.

\begin{proposition}
    There are infinitely many primes.
\end{proposition}

\begin{proof}
    Suppose not and $p_1, \ldots, p_n$ are all the primes. Then, let $p_1\cdots p_n + 1 = N$. 
    
    As we will see, every integer admits a unique decomposition into a product of primes.
\end{proof}

\subsection{Counting Primes}
Let $\pi(x) : N \to \NN$ return the number of primes $p$ such that $0 < p < x$.

Then, $\pi(x)$ is unbounded: $\lim_{x \to \infty} \pi(x) = \infty$. 

\begin{theorem}[Prime Number Theorem]
    \[\lim \frac{\pi(x)}{x / \log x} = 1.\] In other words, $\pi(x) \to \frac{x}{\log x}$>
\end{theorem}

A better approximation is $\mathrm{Li}(x) = \int_{2}^x \frac{dt}{\log t}$. The error for $\mathrm{Li}(x)$ is $|\pi(x) - \mathrm{Li}(x)| = O(\log x \sqrt x)$.

\begin{theorem}[Uniqueness of Prime Factorization]
    Every integer $0 \neq n \in \ZZ$ can be written as \[n = (-1)^{Z(n)} \prod_{p \text{ prime}} p^{a_p} \qquad a_p \in \NN,\] where all but finitely many $a_p$ are zero, $\epsilon(n) = \begin{cases} 0 & n > 0 \\ 1 & n < 0\end{cases}$.
\end{theorem}

To prove this, we first look at a lemma:
\begin{lemma}
    If $a, b \in \ZZ$ and $b > 0$, there exist integers $q, r$ such that $a = qb + r$ and $0 \leq r < b$.
\end{lemma}

\begin{proof}
    Consider the set of integers of the form $\{a - xb | x \in \ZZ\} = S$. The set $S$ contains infinitely many positive integers, so contains a least positive integer $r = a - qb$.

    \begin{remark}
        This property does not hold for $S \subset \QQ$. Consider $S = \{1, \frac{1}{2}, \frac{1}{4}, \ldots\}$. 
    \end{remark}
\end{proof}

The rest of the proof will follow later.

\begin{definition}
    Let $a_1, \ldots, a_n$ be integers. Denote $(a_1, \ldots, a_n)$ to be the set $\{b_1a_1 + \cdots + b_na_n | b_i \in \ZZ\}$.
\end{definition}

\end{document}