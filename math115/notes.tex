\documentclass{article}
\usepackage[utf8]{inputenc}
\usepackage{fancyhdr}
\usepackage{graphicx}
\usepackage[portrait, margin=0.5in, top=0.8in, bottom=0.3in]{geometry}
\usepackage{amsmath, amssymb}
\setlength{\parskip}{1em}
\setlength{\parindent}{0cm} 
\usepackage[mdthm,simplethm]{test}
\usepackage[utf8]{inputenc}
\usepackage{float}
\usepackage{physics}
\usepackage{sectsty}
\usepackage{tikz}
\usepackage{circuitikz}
\allsectionsfont{\normalfont\sffamily\bfseries}

\usepackage{hyperref}
\usepackage{nameref}
\hypersetup{
    colorlinks=true,
    linkcolor=blue,
    filecolor=magenta,      
    urlcolor=cyan,
    pdftitle={Overleaf Example},
    pdfpagemode=FullScreen,
    }

\urlstyle{same}

\setlength{\parindent}{0pt}
\pagestyle{fancy}
\lhead{Math 115 - Introduction to Number Theory}
\rhead{Albert Ye}

\title{Common Core 5th Grade Curriculum}
\author{albert ye}
\date{\today}

\begin{document}
\maketitle

\section{Lecture 1}
\begin{definition}
    An integer $p \neq 0, 1, -1$ is \vocab{prime} if the only integers which divide $p$ are $\pm 1$ and $\pm p$.
\end{definition}

Recall that the integers $\ZZ = \{\ldots, -3, -2, -1, 0, 1, 2, 3, \ldots\}$, $\NN = \{0, 1, 2, 3, \ldots\}$.

\begin{theorem}[Twin Prime Conjecture]
    There are infinitely many $p \in \NN$ such that $p$ is prime and $p + 2$ is prime.
\end{theorem}

Yitang Zhang proved bounded gaps between primes, so there are infinitely many prime $p, p + N$.

\begin{theorem}[Goldbach Conjecture]
    Every even number can be written as the sum of two primes.
\end{theorem}

Vinagradar proved that every odd number can be written as the sum of $3$ primes. The proof should use something called sieves.

\begin{proposition}
    There are infinitely many primes.
\end{proposition}

\begin{proof}
    Suppose not and $p_1, \ldots, p_n$ are all the primes. Then, let $p_1\cdots p_n + 1 = N$. 
    
    As we will see, every integer admits a unique decomposition into a product of primes.
\end{proof}

\subsection{Counting Primes}
Let $\pi(x) : N \to \NN$ return the number of primes $p$ such that $0 < p < x$.

Then, $\pi(x)$ is unbounded: $\lim_{x \to \infty} \pi(x) = \infty$. 

\begin{theorem}[Prime Number Theorem]
    \label{pnt}
    \[\lim \frac{\pi(x)}{x / \log x} = 1.\] In other words, $\pi(x) \to \frac{x}{\log x}$.
\end{theorem}

A better approximation is $\mathrm{Li}(x) = \int_{2}^x \frac{dt}{\log t}$. The error for $\mathrm{Li}(x)$ is $|\pi(x) - \mathrm{Li}(x)| = O(\log x \sqrt x)$.

\subsection{Prime Factorization}
\begin{theorem}[Uniqueness of Prime Factorization]
    \label{uniq}
    Every integer $0 \neq n \in \ZZ$ can be written as \[n = (-1)^{Z(n)} \prod_{p \text{ prime}} p^{a_p} \qquad a_p \in \NN,\] where all but finitely many $a_p$ are zero, $\epsilon(n) = \begin{cases} 0 & n > 0 \\ 1 & n < 0\end{cases}$.
\end{theorem}

To prove this, we first look at a lemma:
\begin{lemma}
    \label{lemma:1}
    If $a, b \in \ZZ$ and $b > 0$, there exist integers $q, r$ such that $a = qb + r$ and $0 \leq r < b$.
\end{lemma}

\begin{proof}
    Consider the set of integers of the form $\{a - xb | x \in \ZZ\} = S$. The set $S$ contains infinitely many positive integers, so contains a least positive integer $r = a - qb$.

    \begin{remark}
        This property does not hold for $S \subset \QQ$. Consider $S = \{1, \frac{1}{2}, \frac{1}{4}, \ldots\}$. 
    \end{remark}
\end{proof}

The rest of the proof will follow later.

\begin{definition}
    \label{multset}
    Let $a_1, \ldots, a_n$ be integers. Denote $(a_1, \ldots, a_n)$ to be the set $\{b_1a_1 + \cdots + b_na_n | b_i \in \ZZ\}$.
\end{definition}

\section{Lecture 2}
\subsection{Prime Factorization, cont.}
Recall the theorem of uniqueness of prime factorizations. Also recall that a prime number $p$ is an integer $\neq 0$, so that the only divisors of $p$ are $\pm 1$ and $\pm p$.

\begin{definition}
    If $0 \neq a \in \ZZ$ and $p \in \ZZ$ is prime, let $\ord_p a$ denote the largest integer $n$ such that $p^n \vert a$, i.e. $a = p^n b$.

    We define $\ord_p 0 = \infty$.
\end{definition}

\begin{lemma}
    \label{lemma:2}
    If $a, b \in \ZZ$, then there exists $d \in \ZZ$ such that $(d) = (a, b)$. Recall Definition \ref{multset} for $(a_1, a_2, \ldots, a_n)$.
\end{lemma}


\begin{proof}
    Let $d$ be the smallest integer $> 0$ in $(a, b)$. We claim that $(d) = (a, b)$. As $d \in (a, b)$, we see that $(d) \subseteq (a, b)$. We have to show that $(a, b) \subseteq (d)$.

    Take $c \in (a, b)$, then we see from  \ref{lemma:1} that $c = qd + r$ with $0 \leq r < d$. THen $r = c-qd \in (a,b)$. By minimality of $d$, we see that $r = 0$, so $c = qd$ implie $c \in (d)$.
\end{proof}

\begin{definition}
    If $a, b \in \ZZ$, then a greatest common divisor $d$ of $a, b$ is an integer which divides $a, b$ such that any other integer $c$ with that property satisfies $c | d$.
\end{definition}

\begin{remark}
    If we insist $d \geq 0$, then it is unique. Because if $c, d \geq 0$ are both $\gcd(a, b)$, then $c | d$ and $d | c$, which implies $c = \pm d$, but because of positivity we must have $c = d$.
\end{remark}

\begin{proposition}
    If $a, b \in \ZZ$, then the $d$ appearing in \ref{lemma:2} s.t. $d = (a, b)$ is a greatest common divisor of $a, b$.
\end{proposition}

\begin{proof}
    If $(d) = (a, b)$, then $a \in (d) = d\ZZ \implies d | a$. If $c \in \ZZ$ is any common divisor of $a$ and $b$, then $c$ divides $an + bm$ for all $m, n \in \ZZ$. As $d \in (a, b)$, $d$ has this form, so $c | d$.

    Thus, by definition, $d$ must be the greatest common divisor.
\end{proof}

\begin{definition}
    We say that $a, b \in \ZZ$ are \vocab{relatively prime} if $(a, b) = 1$.

    In other words, the only nonzero integers that divide $a$ and $b$ are $\pm 1$.
\end{definition}

\begin{lemma}
    \label{lemma:3}
    Suppose $a | bc$, and $(a, b) = 1$. Then, $a | c$.
\end{lemma}

\begin{proof}
    $(a, b) = 1$ implies $1 = an + bm$ for some $n, m$. So $c = acn + bcm$. Notice that the right term contains $bc$ and the left term contains $a$, so $c$ must be divisible by $a$.
\end{proof}

\begin{corollary}
    \label{cor:3}
    If $p$ is prime and $p | ab$, then $p | a$ or $p | b$.
\end{corollary}

\begin{proof}
    If $(p, a) = p$, then we're done as $p | a$.

    Suppose instead that $(p, a) = 1$. From \ref{lemma:3}, we have $p | b$.
\end{proof}

We take the contrapositive to see that if a prime $p$ doesn't divide $a$ or $b$, then it doesn't divide $ab$.

\begin{proposition}
    Fix a prime $p$. If $a, b \in \ZZ$, then $\ord_p ab = \ord_p a + \ord_p b$.
\end{proposition}

\begin{proof}
    Let $\ord_p a = n, \ord_p b = m$. Then, we see that $a = p^nc, b = p^md$ where $p \not \vert c, p \not \vert d$. So $ab = p^nc \cdot p^md = p^{n+m}(cd)$. We know that $p$ cannot divide $cd$ from \ref{cor:3}, so $\ord_p ab = n + m$.
\end{proof}

Now, we can finally prove Theorem \ref{uniq}.
\begin{proof}[Proof of \ref{uniq}]
    Fix $n \in \ZZ$ and suppose that $n = (-1)^{\epsilon(n)} \prod_{p} p^{a_p}$.

    Then, fix a prime $q$. We see that \[\ord_{q} n = 0 + \sum_{p}a_p \ord_q p = a_q.\] This is because $\ord_q p = \begin{cases}1 & q = p \\ 0 & q \neq p\end{cases}$. This implies that the only factors that will contribute to $\ord_q n$ are the terms of $q$, of which there are $a_q$.

    Hence, $a_p$ for each prime $p$ is determined solely by $n$, so the prime factorization is unique.
\end{proof}

\section{Lecture 3}

\begin{lemma}
    \label{lemma:3:1}
    Every nonconstant irreducible polynomial has a factorization into nonconstant irreducible polynomials.
\end{lemma}

\section{Lecture 4}
\subsection{Factorization of Polynomials}
Recall \ref{lemma:3:1} from last lecture.

Again let $k = \QQ, \RR, \CC$. 

\begin{definition}
    A nonzero polynomial is called \vocab{monic} if the coefficient of its leading term is $1$.
\end{definition}

\begin{definition}
    If $p(x) \in k[x]$ is nonconstant irreducible, and $0 \neq q(x) \in k[x]$ is any other polynomial. Let $\ord_p q$ be defined as the greatest integer $n \geq 0$ such that $p^n(x) \vert g(x)$ but $p^{n+1}(x) \not \vert g(x)$.
\end{definition}

\begin{theorem}
    \label{polyfac}
    Every nonconstant polynomial $g(x)$ admits a unique factorization of the form $g(x) = c \prod_{p(x)} p(x)^{a_p}$, where $c \in k^x = k \setminus \{0\}$ and the product is over all irreducible, nonconstant, monic polynomials.

    Then, $a_p = \ord_p g$, and $c$ is the leading term of $g$.
\end{theorem}

We start with the following lemma: 

\begin{lemma}
    If $f(x), g(x) \in k[x]$ are polynomials with $0 \neq g(x)$ then we can find polynomials $q(x)$ and $r(x)$ with either $r(x) = 0$ or $0 \leq \deg r(x) < \deg g(x)$ s.t. $f(x) = q(x)g(x) + r(x)$.
\end{lemma}

\begin{proof}
    If $g|f$, then $g(x)q(x) = f(x)$ for some $q(x)$, and let $r(x) = 0$. Suppose otherwise, and $f \neq 0$. Consider the set $f(x) \in \{f(x) - h(x)g(x), h(x) \in k[x]\}$, and let $q(x)$ be such that $r(x) = f(x) - q(x)g(x)$ is of least degree in this set. 

    It remains to show $r = 0$ or $\deg r < \deg g$. Suppose otherwise, and that $r(x)$ has leading term $ax^d$ and $g(x)$ has leading term $bx^n$ with $d \geq n$. Let $m9x) = \frac{a}{b} x^{d - n} g(x)$. Then $m(x)$ is a polynomial such that $\deg(r(x) - m(x)) < \deg r(x)$. 

    However, $r(x) - m(x) = f(x) - (q(x) + \frac{a}{b} x^{d - n}) g(x)$, so $r(x) - m(x) \in S$. This contradicts the definitions of $r(x)$.
\end{proof}

\begin{definition}
    If $f_1(x), \ldots, f_n(x)$ are polynomials, let $(f_1, f_2, \ldots, f_n)$ be defined similarly to integers.
\end{definition}

\begin{lemma}
    Given $f(x), g(x) \in k(x)$, there is a $d(x) \in k[x]$ s.t. $(f, g) = (d)$. 
\end{lemma}

\begin{proof}
    Let $d(x)$ be a polynomial of least degree in $(f, g)$. We have $(d) \subset (f, g)$. Let $c(x) \in (f, g)$. Then, if $d \vert c$, we're done. If not, then there exists $q(x), r(x)$ s.t. $c(x) = q(x)d(x) + r(x)$, with $\deg r(x) < \deg d(x)$. Then $r(x) = c(x) - q(x)d(x) \in (f, g)$, which is a contradiction as $\deg r < \deg d$. 
\end{proof}

\section{Lecture 5}
Continue proving \ref{polyfac}.

\begin{definition}
    We say $f(x), g(x) \in k[x]$ are \vocab{relatively prime} if $(f, g) = 1$.
\end{definition}

\begin{definition} 
    A greatest common divisor, or $\gcd$ of $f$ and $g \in k[x]$ is a polynomial $d(x)$ which divides $f$ and $g$ and has the property that if $c(x) \in k[x]$ divides $f$ and $g$ then $c | d$. (Ambiguous up to a scalar.)
\end{definition}

\begin{lemma}
    \label{poly:div}
    If $f$ and $g$ are relatively prime and $f | gh$, then $f | h$. 
\end{lemma}

\begin{proof}
    If $(f, g) = 1$ then $1 = a(x)f(x) + b(x)g(x)$. So $h(x) = a(x)f(x)h(x) + b(x)g(x)h(x) = f(x)(a(x)h(x) + b(x)j(x))$ for some other polynomial $j(x)$. Then, $f(x) | h(x)$.
\end{proof}

If $d(x) = (f(x), g(x))$ and $x \in k^x$ then $\alpha d$ is also a gcd o $f$ and $g$; $(\alpha d) = (d)$. 

Now, recall that a nonconstant polynomial $f(x)$ is \vocab{irreducible} if its only divisors are of the form $\alpha f$ or $\alpha$ ($\alpha \in k^*$); i.e. if any polynomial divides $f$, it's either a scalar or a scalar multiple of $f$.

\begin{lemma}
    If $p(x)$ is irreducible and $p | fg$, then $p | f$ or $p | g$. 
\end{lemma}

\begin{proof}
    $(p, f) = (1)$ or $(p) = (\alpha p)$ for all $x \in k^*$. If $(p, f) = (p)$, then $p | f$. Otherwise, $(p, f) = (1)$, so from Lemma \ref{poly:div} we have $p | g$. 
\end{proof}

\begin{definition}[Order in Polynomial Terms]
    If $p$ is a nonconstant polynomial and $g \neq f \in k[x]$ then $\ord_p f$ is the largest $a \in \ZZ_{\geq 0}$ such that $p^a | f$. 
\end{definition}

\begin{lemma}
    If $p(x) \in k[x]$ is irreducible and $a, b \in k[x]$, then $\ord_p(ab) = \ord_p(a) + \ord_p(b)$. 
\end{lemma} 

Finally, we can prove \ref{polyfac}. 

\begin{proof}
    Weite $0 \neq f(x) = c \prod_p p(x)^{a_p}$. For every monic irreducible polynomial $q$, $\ord_q f = \sum_f a_p \ord_q p$, and we see that $\ord_q p = \begin{cases}1 & q = p \\ 0 & q \neq p.\end{cases}$. This must be $a_q$.

    The scalar $c$ is the leading coefficient of $f$, so every polynomial factorization uniquely determines one polynomial. 
\end{proof}

\section{Lecture 6}
\begin{proposition}
    If $k = \QQ, \RR, \CC$ (any field) then $k[x]$ contains infinitely many irreducible polynomials.
\end{proposition}

\begin{proof}
    Suppose not, and $p_1(x), \ldots, p_n(x)$ exhaust the irreducible polynomials. Thus $q(x) = 1 + p_1(x)p_2(x) \cdots p_n(x)$ is a polynomial not divisible by the $p_i(x)$, but it must factor into a product of the $p_i(x)$, a contradiction. 
\end{proof}

\begin{lemma}
    Every integer $n \neq 0$ can be written as $n = ab^2$ where $a$ is squarefree.
\end{lemma}

\begin{definition}
    An integer $n \neq 0$ is squarefree if it isn't divisible by the square of any prime.
\end{definition}

\begin{proof}
    If $|n| = 1$ then it's squarefree. If $|n| > 1$ then $n = (-1)^{\epsilon(n)} p_1^{2a_1 + b_1} \cdots p_m^{2a_m + b_m}$, where $b_i$ is either $0$ or $1$ for all $i$. Then, in turn, \[n = [p_1^{2a_1}\cdots p_m^{2a_m}][(-1)^{\epsilon(n)} p_1^{b_1}\cdots p_m^{b_m}].\] We see that the first term is $b^2$ and the second term is a squarefree $a$.
\end{proof}

\begin{definition}
    $\nu(n) = $number of positive divisors 

    $\sigma(n) = $sum of positive divisors 
\end{definition}

\begin{proposition}
    Let $n \in \ZZ_{> 1}$ have a prime factorization $n = p_1^{a_1} \cdots p_m^{a_m}$. Then,
    \begin{itemize}
        \item $\nu(n) = (a_1 + 1)(a_2 + 1)\cdots(a_n + 1)$
        \item $\sigma(n) = \left(\sum_{i = 0}^{a_1} p_1^i\right) \cdots \left(\sum_{i = 0}^{a_n} p_n^i\right)$.
    \end{itemize}
\end{proposition}

Recall that $\sum_{n = a}^b x^n = \frac{x^{b + 1} - x^a}{x - 1}$, so $\sigma(n) = \left(\frac{p_1^{a_1 + 1} - 1}{p_1 - 1}\right) \cdots \left(\frac{p_n^{a_n + 1} - 1}{p_n - 1}\right)$.

\begin{definition}
    An integer $> 0$ is \vocab{perfect} if $\sigma(n) = 2n$.
\end{definition}

Euler claimed that every even perfect number can be written as $2^m(2^{m + 1} - 1)$, where $2^{m + 1} - 1$ is a Mersenne prime.

\subsection{Mobius Function}

\begin{definition}[Mobius Mu Function]
    The Mobius $\mu : Z_{> 0} \to \{0, \pm 1\}$ returns $\mu(n) = 0$ if $n$ is not squarefree, $\mu(1) = 1$, and if $n > 1$, $n = p_1, \ldots, p_m$, then $\mu(n) = (-1)^m$.
\end{definition}

\begin{proposition}
    If $n > 1$ then $\sum_{d | n} \mu(d) = 0$.
\end{proposition}

\begin{proof}
    $n = p_1^{a_1} \cdots p_m^{a_m}$. Notice that for any $a_i > 1$, we can ignore and take mod $2$ because non-squarefree implies a Mobius of $0$.
    
    Therefore, $\sum_{d|n} \mu(d) = \sum \mu(p_1^{\epsilon_1} \cdots p_m^{\epsilon_m}) = (1 - 1)^m = 0$.
\end{proof}

\subsection{Dirichlet Convolution}

\begin{definition}
    If $f, g$ are two functions $\ZZ_{> 0} \to \CC$, then the Dirichlet convolution of $f$ and $g$ is defined to be $(f \cdot g)(n) = \sum_{d | n} f(d)g(\frac{n}{d})$.
\end{definition}

\begin{remark}
    Dirichlet convolution is associative; given $f, g, h : \ZZ_{> 0} \to \CC$, then $((f \cdot g) \cdot h)(n) = (f \cdot (g \cdot h))(n) = \sum f(d_1) g(d_2) h(d_3)$,
\end{remark}

\begin{definition}
    Let $1(n) = \begin{cases}1 & n = 1 \\ 0 & n > 1\end{cases}$. Then, $(f * 1)(N) = \sum_{d | n} f(d)$. 
\end{definition}

\begin{theorem}[Mobius Inversion]
    If $f : \ZZ_{> 0} \to \CC$ and $F(n) = \sum_{d | n) f(d)}$, then $\sum_{d|n} F(d) \mu\left(\frac{n}{d}\right) = f(n)$, or as we simplify it, $\mu \times F = f$.
\end{theorem}

\section{Lecture 7}
\subsection{Prime Counting}
\begin{definition}[Euler Totient]
    We define $\phi: \ZZ_{> 0} \to \ZZ_{> 0}$. $\phi(n)$ is the number of integers in $[1, n]$ relatively prime to $n$. 

    $\phi(1) = 1$, $\phi(p) = p-1$ for prime $p$.
\end{definition}

\begin{proposition}
    $(\phi \cdot )(n) = \sum_{d | n} \phi(d) = n$.
\end{proposition}

\begin{proof}
    Consider the set $\left\{\frac{1}{n}, \frac{2}{n}, \ldots, \frac{n}{n}\right\}$. Write these fractions in lowest terms. 
    
    For each $d | n$, we wish to count the functions above with $d$ in lowest terms. These fractions will be a subset of the fractions $\frac{a}{n}$ where $\frac{n}{d} | a$, i.e. a subset of the fractions $\left\{\frac{1}{d}, \frac{2}{d}, \ldots, \frac{d}{d}\right\}$. There are $\phi(d)$ many fractions on this list with $d$ in the domain, when written in lowest terms.

    So if $J_d \subset \left\{\frac{1}{n}, \frac{2}{n}, \ldots, \frac{n}{n}\right\}$ corresponds to the fractions of denominator $d$ in lowest terms, then $S = \bigcup_{d | n} J_d$, and $n = |S| = \sum_{d | n} |J_d| = \sum_{d | n} \phi(d)$. 
\end{proof}

From Mobius inversion, we have $\phi = (\phi \cdot 1) \cdot \mu$. We know that $(\phi \cdot 1) = id$ where $id(n) = n$, so we have $\mu \cdot id = \sum_{d | n} \mu(d) \frac{n}{d}$. Now, let $n = p_1^{a_1}\cdots p_m^{a_m}$.  Then,
\begin{align*}
    \mu \cdot id &= n - \sum_{i} \frac{n}{p_i} + \sum_{i < j} \frac{n}{p_ip_j} - \sum_{i < j < k} \frac{n}{p_ip_jp_k} \cdots \text{ (by definition of Mobius inversion)}\\
    &= n\left(1 - \frac{1}{p_1}\right)\left(1 - \frac{1}{p_2}\right)\cdots \left(1 - \frac{1}{p_m}\right) = \phi(n).
\end{align*}

\begin{theorem}
    \label{8:1}
    $\sum_{p \text{ prime}} \frac{1}{p}$ diverges.
\end{theorem}

Also consider $\pi(x) = \frac{x}{\log x}(1 + \left(\frac{1}{\log x}\right)$.

\begin{proof}
    Of $n \in \ZZ_{> 0}$, let $p_1, \ldots, p_{\pi(n)}$ be the primes $\leq n$ and let \[\lambda(n) = \prod_{i = 1}^{\pi(n)} \left(1 - \frac{1}{p_i}\right)^{-1}.\] Notice that each inner value for the product term is $\sum_{a = 0}^\infty \left(\frac{1}{p_i}\right)^a$. 

    Then, $\lambda(n) = \sum \frac{1}{p_1^{a_1} \cdots p_{\pi(n)}^a}$, where the sum is over all $\pi(n)$-tuples $(a_1, \ldots, a_{\pi(n)}) \in \ZZ_{\geq 0}^{\pi(n)}$. Then, we have \[\log \lambda(n) = -\sum_{i = 1}^{\pi(n)} \log(1 - p_i)^{-1} = \sum_{i = 1}^{\pi(n)} \sum_{m = 1}^\infty (mp_i^m)^{-1}.\] If we can prove that $\log \lambda(n)$ converges, then we see that $\lambda(n)$ is divergent and we are done. 

    I'll pick this up later.

    Somehow we're done. Easy.
\end{proof}

\section{Lecture 8}
We go back to \ref{8:1}. Because of a fire alarm, there wasn't anything else covered.

\section{Lecture 9}
\subsection{Estimates for Prime Counting Function}

Last time, we proved that $\sum_{p} \frac{1}{p}$ for prime $p$ diverges. 

We go back to \ref{pnt}, the Prime Number Theorem.

\begin{theorem*}[Prime Number Theorem]
    \[\pi(x) = \frac{x}{\log x} \left(1 + O\left(\frac{1}{\log x}\right)\right).\]
\end{theorem*}

\begin{theorem}
    \[\sum_{p \text{ prime}, p \leq n} \frac{1}{p} = \log \log n + C.\]
\end{theorem}

\begin{proof}
    Sketch: We turn $p$ into a function of $\pi(x)$, and then use $\pi(n)$
\end{proof}
We let $\theta(x) = \sum_{p \leq x, p \text{ prime}} \log p$. 

\section{Lecture 10}
\subsection{Estimates for Prime Counting Function}
\begin{lemma}
    $\theta(x) = \sum_{p \leq x, p \text{ prime}} \log p < (4 \log 2) x$. 
\end{lemma}

\begin{proposition}
    There exists a constant $C_2 \in \RR_{> 0}$ such that $\pi(x) < \frac{c_2 x}{\log x}$.
\end{proposition}

\begin{proof}
    \begin{align*}
        \theta(x) &= \sum_{p \leq x, p \text{ prime}} \log p \\
        &> \sum_{p \geq \sqrt x, p \text{ prime}}^{p \leq x} \log p \\
        &\geq \log \sqrt x (\pi(x) - \pi(\sqrt x)) \\
        &= \log \sqrt x \pi(x) - \sqrt x \log \sqrt x.
        \\ 
        \pi(x) &\leq \frac{2\theta(x)}{\log x} + \sqrt x \\
        &< (8 \log 2) \frac{x}{\log x} + \sqrt x \\
        &< (2 + 8 \log 2) \frac{x}{\log x} < \frac{2x}{\log x}.
    \end{align*}

    Therefore, we see that $c_2 \geq 2$ works. 
\end{proof}

\begin{proposition}
    There exists a constant $c_1 \in \RR \geq 0$ such that $\frac{c_1 x}{\log x} < \pi(x)$.
\end{proposition}

\begin{proof}
    $\theta(x) \sim \dbinom{2n}{n} = \left(\frac{n+1}{1} \right)\left(\frac{n+2}{2} \right)\cdots\left(\frac{2n}{n} \right) \geq 2^n$.

    Let $t_p = \lfloor \frac{\log 2n}{\log p} \rfloor = \log_p 2n.$
    We also see that $n \log 2 \leq \sum_{p \leq n, p \text{ prime}} t_p \log p = \sum_{p \leq n, p \text{ prime}} \lfloor \frac{\log 2n}{\log p} \rfloor \log p = K$.

    If $\log p > \frac{1}{2} \log (2n)$, then $\frac{\log 2n}{\log p} < 2$ and its floor is $1$.

    Then, $K = \sum_{p \text{ prime}, p \leq \sqrt{2n}} \lfloor \frac{\log 2n}{\log p}\rfloor \log p + \sum_{p\text{ prime}, 2n > p > \sqrt{2n}} \log p \leq \theta(2n)$.

    Putting in $n \log 2 \leq K$, we have $n \log 2 \leq \sqrt{2n}\log 2n + \theta(2n)$, so $\theta(2n) \geq n \log 2 - \sqrt{2n} \log 2n$.

    Next time, we'll show that this estimate for $\theta$ implies a lower bound for $\pi(x)$.
\end{proof}

\section{Lecture 11}
\subsection{Announcements}
\label{sec.11.1}
Brief summary of topics covered:

1.1, 1.2, entirety of Chapter 2, 3.1, 3.2. The hardest parts are also not easy to test, and should not be memorized. Because that would be absolutely fucking evil.

\subsection{Estimates for Prime Counting Function}
Last time, we were trying to prove:
\begin{theorem}
    \[\pi(x) < \frac{c_2}{\log x}, \] for some constant $c_2 \in \RR$. 
\end{theorem} 

\subsection{Congruence}
We define $a \equiv b \pmod n$ to mean that $n | (a - b)$.

\begin{definition}
    The relation of congruence defines an \vocab{equivalence relation} $a R b$. An equivalence relation is a relation which is revlexive, symmetric, and transitive.

    \begin{itemize}
        \item $a \equiv a$
        \item $a \equiv b \implies b \equiv a$
        \item $a \equiv b, b \equiv c \implies a \equiv c$.
    \end{itemize}
\end{definition}

Given a set $S$ and an equivalence relation $R$, on $S$ we can form the set $[x] = \{y \in S | y R x\}$. 

In our case, $a \in \ZZ \implies \overline a := [a] = \{b \in \ZZ | a \equiv b \pmod n\}$.

\begin{definition}
    If $a \in \ZZ$, $\overline a = \{a + nb | b \in \ZZ\}$ is called the \vocab{congruence class} of $a$ for the modulus $n$.
\end{definition}

\begin{proposition}
    The equivalence classes for $R$ on $S$ partition $S$.

    That is, every $x \in S$ is in some equivalence class $(x \in [k])$, and given $x, y \in S$, either $[x] = [y]$ or $[x] \cap [y] = \emptyset$.
\end{proposition}

\begin{proof}
    If $z \in [x] \cap [y]$, $x R z$ and $z R y$ so $x R y$, so any $w \in S$ has the property that $wRx = wRy \implies [x] = [y]$.
\end{proof}

For congruence, this boils down to the fact that 
\begin{itemize}
    \item $\overline a = \overline b \iff a \equiv b \pmod n$.
    \item $\overline a \neq b \iff \overline a \cap \overline b = \emptyset$
    \item There are precisely $n$ congreunce classes modulo $n$.
\end{itemize}

\section{Lecture 12} 
\subsection{Equivalence Classes, Continued}
The set of equivalence classes modulo $n$ is defined as $\ZZ / n\ZZ$. 

\begin{lemma}
	The set $\ZZ /n\ZZ$ admits addition and multiplication by the formulas:
	\begin{itemize}
		\item $\overline{a} + \overline{b} = \overline{a + b}$.
		\item $\overline{a} \cdot \overline{b} = \overline{ab}$.
	\end{itemize}
\end{lemma}

\begin{proof}
	To check that $+$ and $\cdot$ are defined in this way are well-defined, we must show that we'd get the same thing when changing the representatives, i.e., replacing $\overline a$ by $a + kn$ for $k \in \ZZ$. 

	\[\overline{a + kn + b + jn} = \overline{a + b} \implies a + kn + b + jn \equiv a + b \pmod n,\] which is true after removing all obvious multiples of $n$. The proof for $\overline{ab}$ is too long to fit in the margins, but it works similarly.
\end{proof}

An application of this is finding which polynomials in $\ZZ[x]$ have no integer solutions. 


\begin{remark}
	If $a \equiv b \pmod n$, then $a^m \equiv b^m \pmod n$.

	Let $p(x) = c_mx^m + \cdots + c_1x + c_0$ for integer $c_i$. If $a \equiv b \pmod n$, we further see that $p(\overline a) := \overline{p(a)}$ must equal $p(\overline b)$. Therefore, $c_ma^m \equiv c_mb^m \pmod n$, so $\overline{c_m}a^m = \overline{c_m}b^m$.
\end{remark}

Now, we suppose $n = 2$. Then, $p(0) = c_0$, $p(1) = c_n + c_{n-1} + \cdots + c_0$. If $p(x)$ has a solution in the integers, then it must have a solution mod $2$ ($p(k) = 0\implies p(\overline k)  \overline 0$. So, any $p(x) \in k[x]$ with integer solutions must have $p(\overline 0) = \overline 0$ or $p(\overline 1) = \overline 1$. 
\begin{proposition}
	Any $p(x)$ with $c_0$ odd and $\sum_{i = 0}^n c_i$ odd has no integer solutions. 
\end{proposition}

Now, for the general criterion modulo $n$:

\begin{theorem}
	If $c_0 \not \equiv 0 \pmod n$, and $\sum_{i = 0}^m k^i c_i \not \equiv 0 \pmod n$ for all $0 < k < n$, then $p(x)$ has no integer solutions.
\end{theorem}

\section{Lecture 13}
\subsection{Announcements}
Topics of the test: same as in \ref{sec.11.1}.

Need to know the facts about divergence $\sum_{p} \frac{1}{p}$, $\frac{c_1x}{\log x} < \pi(x) < \frac{c_2x}{\log x}$, but don't need to remember the whole proofs for all of them.

Also there is another book with a lot of review problems: Niven, Montgomery, and Zuckerman, \textit{An Introdution to the Theory of Numbers}, 5e. §1.2, §1.3, §4.2, §4.3 are good practice. If not much time, because of tests like the 126 midterm, just review and understand solutions for the questions assigned. 

\subsection{Diophantine Equations}
We're looking for solutions to $ax \equiv b \pmod m$. Because if $x_0$ solves $ax_0 \equiv b \pmod m$, then so does $x_0 + km$.

\begin{lemma}
	Let $d := (a, m)$ and let $a' = \frac{a}{d}$, $m' = \frac{m}{d}$.
	The equation $ax \equiv b \pmod m$ admits a solution iff $d | b$. 

	If $d|b$ then there are exactly $d$ solutions.

	If $x_0$ is a solution, tnen $x_0 + m', x_0 + 2m', \ldots, x_0 + (d-1)m'$ is a list of all solutions to $ax \equiv b \pmod m$.
\end{lemma}

\begin{proof}
	For the first claim: suppose we've found a solution $x_0$ to $ax \equiv b \pmod m$, i.e. $m | (ax_0 - b)$. So, $ax_0 = b \equiv km$ for source $k \in \ZZ$. As $d | a$ and $d | m$, $d| b$. Conversely, if $d | b$, then set $c = \frac{d}{b}$. Because $d = (a, m)$ we can find $x_0', y_0' \in \ZZ$ such that $ax_0' + my_0' = d$. Multiplying by $c$, we have $cax_0' + cmy_0' = b$, so $x_0 = cx_0'$ solves $ax_0 \equiv b \pmod m$. \qed 

	For the second claim: suppose $x_0$ and $x_1$ both solve $ax \equiv b \pmod m$. So, $ax_0 - b = k_1 m$ and $ax_1 - b \equiv k_2 m$. Then, this means that $a(x_0 - x_1) \equiv 0 \pmod m$. $d = (a, m)$, so $m | a(x_0 - x_1)$ implies $\frac{m}{d} = m'$ divides $x_0 - x_1$, so $x_1 = x_0 + km'$ for some $k \in \ZZ$. 

	If$x_0$ solves $ax \equiv b \pmod m$, then $x_0 + km'$ does too for every $k \in \ZZ$. $akm'$ is divisible by $m$ so they must be $\equiv 0 \pmod m$, so $ax_0 \equiv a(x_0 + km') \pmod m$. 

	We claim that $x, y \in x_0 , x_0 + m', \ldots, x_0 + (d-1)m'$ are inequivalent modulo $m$, which follows as $(x_0 + k_1m') - (x_0 + k_2m') = (k_1 - k_2) m'$ and $0 < k_1 - k_2 < d$, since $0 \leq k2 < k_1 < d$. Therefore, this is not equivalent to $0$ modulo $m$. \qed

	For our third claim: if $x_0 + km'$ is any solution to $ax \equiv b \pmod m$, hen $k \equiv k' \pmod d$ for some $k' \in [0, d)$. Then, we have $km' \equiv k'm' \pmod m$, so $x_0 + km' \equiv x_0 + k'm' \pmod m$.
\end{proof}

\begin{corollary}
	if $(a, m) = 1$, the equation $ax \equiv b \pmod m$ has exactly one solution.
\end{corollary}

\begin{corollary}
	If $m = p$ is prime, then $ax \equiv b \pmod p$ has only one solution provide $p \not\vert a$. 
\end{corollary}

\begin{remark}
	If $m$ is not prime, then $m = m_1 m_2$ with $0 < m_1, m_2 < m$ so $m_1 m_2 \equiv 0 \pmod m$. So in $\ZZ / m\ZZ$, $\overline{m_1} \neq 0 \neq \overline{m_2}$ but $\overline{m_1} \overline{m_2} = 0$. 
	Then, $\overline m_1$ cannot have an inverse $\overline{m_1}^{-1}$ because then $\overline{m_2} = \overline{m_1}^{-1} \overline{m_1} \overline{m_2} = \overline{m_1}^{-1} \overline{0} = 0$.
\end{remark}

\section{Lecture 14}
\subsection{Inverse}
The main question for today is when $a \in \ZZ / n \ZZ$ has a multiplicative inverse. 

\begin{theorem}[Euler's Theorem] 
	if $m > 1$ is an integer an $(a, m) = 1$, then $a^{\phi(m)} \equiv 1 \pmod m$. 
\end{theorem}

We also have a corollary: 
\begin{corollary}
	If $p$ is prime and $p$ doesn't divide $a$, then $a^{p-1} \equiv 1 \pmod p$. 
\end{corollary}

This corollary is Fermat's little theorem.

\begin{proof}
	Form the set $S$ of units in $\ZZ / m \ZZ$. $|S| = \phi(m)$ and $a \in S$ as $(a, m) = 1$. We claim that $s = \{as | s \in S\}$. 

	We see that $S$ is bijective to $S$, so we see that $S$ is bijective to $aS$, as $a$ is relatively prime to $m$. We see that \[\prod_{s \in S} s = \prod_{s \in S} as = a^{|S|} a^{\phi(m)} \prod_{s \in S} s.\] Taking the inverse, we see that $a^{\phi(m)} \equiv 1 \pmod m$. 
\end{proof}

\subsection{Chinese Remainder Theorem}
\begin{theorem}[Chinese Remainder Theorem]
	Suppose that $m = m_1, \ldots, m_t$ and that $(m_i, m_j) = 1$. For all $1 \leq i < j \leq t$, let $b_1, \ldots, b_t \in \ZZ$ and consider the system of congruences \begin{align*}x &\equiv b_1 \pmod{m_1},\\ x &\equiv b_2 \pmod{m_2}, \\ &\ldots \\ x &\equiv b_t \pmod{m_t}.\end{align*} 
	This system has a solution and any two solutions differ by a multiple of $m$.  
\end{theorem}

That is, any system of congruences centered around $x$ must have a unique $x$ modulo $m$, where $m$ is the product of all moduli in the system, and all moduli in the system are coprime.

\begin{lemma}
	If $m \in \ZZ$ and $a_1, \ldots, a_t$ are each relatively prime to $m$, then $a-1, \ldots, a_t$ is. 
\end{lemma}

\begin{lemma}
	If $a_1, \ldots, a_t$ divide $n \in \ZZ$ and $(a_i, a_j) = 1$ for $1 \leq i < j \leq t$, then $a_1 \cdots a_t | n$.
\end{lemma}

\begin{proof}
	The integers $a_i$ determine a partition of a subset of the prime factors of $n$, by definition.
\end{proof}

I honestly tuned out the proof of CRT oops

\section{Lecture 15}
161 

\section{Lecture 16}
\subsection{The Primitive Element}
\begin{theorem}[Primitive Element] 
	\label{16.1}
	For each prime $p$, there is an $\alpha \in \ZZ / p\ZZ$ s.t. $(\ZZ / p\ZZ)^* = \{1, \ldots, p + 1\} = \{\alpha, \alpha^2, \ldots, \alpha^{p+1}\}$.
\end{theorem}

\begin{proposition}
	If $d | p - 1$, then $x^d \equiv 1 (p)$ has exactly $d$ solutions in $\ZZ / p\ZZ$.
\end{proposition}

We are finally making it into the abstract algebra class with this next one:
\begin{definition}[Group] 
	A \vocab{group} is a set $S$ endowed with a binary operation $\cdot$, with the property that $x \cdot y \in S$

	\begin{enumerate} 
		\item If $x, y, z \in S$, $(x \cdot y) \cdot z = x \cdot (y \cdot z)$. \textit{(associativity)}
		\item There is an element $e \in S$ s.t. $e \cdot x = x = x \cdot e$ for all $x \in S$. \textit{(identity)}
		\item For any $x \in S$ there exists $x^{-1} \in S$ s.t. $x \cdot x^{-1} = e = x^{-1} \cdot x$. \textit{(invertibility)}
	\end{enumerate}
\end{definition}

\begin{definition}[Cyclic Group] 
	A group is \vocab{cyclic} if there exists an element $g \in G$ such that every $x \in G$ equals $g^i$ for some $i$ depending on $x$. 
\end{definition}

So, Theorem \ref{16.1} essentially wants us to prove that $((\ZZ / p\ZZ), \times)$ forms a cyclic group!

\begin{definition}[Order] 
	If $x \in G$ and $G$ is a inite group, then the order $\ord x$ of $x$ is the least positive integer such that $x^{\ord x} = e$.

	$\ord e = 1, \ord x > 1$ if $x \neq e$.
\end{definition}

\begin{lemma} 
	If $G$ is a finite abelian group of order $n$, then if $x \in G, x^n = e$.
\end{lemma}

\begin{corollary}
	If $G$ is finite (abelian) and $x \in G$, then $\ord x | |G|$.
\end{corollary}

\begin{proof}[Proof of Primitive Element]
	If $d | p-1$, let $\psi(d)$ denote the number of elements of $(\ZZ / p\ZZ)^*$ of order $d$. An element $a \in (\ZZ / p\ZZ)^*$ satisfies $a^d = 1$ if and only if $\ord a | d$ If $d = c \cdot \ord a$, then $a^d = (a^{\ord a})^c = 1^c = 1$. Conversely, if $a^d = 1 \in \ZZ/p\ZZ$, then I claim $\ord a | d [a^{(\ord a, d)} = 1$.

	From the proposition, we see that $d = \sum_{c | d} \psi(c)$. Taking the Mobius inversion, we get that $psi = \phi$, so there is an element of $(\ZZ / p\ZZ)$ with order $p-1$. \qed
\end{proof}

\section{Lecture 17}
Reminder that not all $n$ are cyclic; example is $n = 4$.

\begin{theorem}
	\label{17.1}
	If $p$ is an odd prime and $l \in \ZZ_{> 0}$, $(\ZZ / p^l \ZZ)^*$ possess a primitive element, so it is cyclic. 
\end{theorem} 

\begin{lemma}
	If $p$ is prime and $1 \leq k \leq p$, then $\dbinom{p}{k}$ is divisible by $p$.
\end{lemma} 

\begin{proof} 
	$\dbinom{p}{k} = \dfrac{p!}{k!(p-k)!}$, and $k < p, p-k < p$ so $p$ does not divide $k!, (p-k)!$. Thus, there must be a factor of $p$.
\end{proof}

\begin{lemma}
	If $a \equiv b \pmod{p^l}$, then $a^p \equiv b \pmod{p^{l+1}}$. 
\end{lemma}

\begin{proof}
	$a = b + cp^l$, so \begin{align*}
		a^p &= (b+cp^l)^p \\
		&= b^p + \binom{p}{1}b^{p-1} cp^l + A
	\end{align*}
	where $p^{2l} | A$. As $2l \geq l + 1$, $a^p \equiv b^p \pmod{p^{l+1}}$.
\end{proof}

\begin{corollary}
	If $l \geq 2$ and $p \neq 2$, then $(1 + ap)^{p^{l-1}} \equiv 1 + ap^{l-1} \pmod{p^l}$ for all $a \in \ZZ$.
\end{corollary}

\begin{proof}
	induct on $l$. If $l = 2$, then $(1 + ap)^1 = 1 + ap^1$.

	Assume it holds for some $l \geq 2$, we prove it for $l + 1$.

	\begin{align*}
		(1 + ap)^{p^{l-1}} &= ((1 + ap)^{p^{l-2}})^p.
	\end{align*}

	From Lemma $2$, we have thta this is congruent to $(1 + ap^{l-1})^p$ modulo $p^{l + 1}$. 
	And then, $(1 + ap^{l-1})^p = 1 + \binom{p}{1}ap^{l-1} + B$, so all the terms in $B$ are divisible by $p^{1 + 2(l-1)}$, except for the last term $a^p p^{p(l-1)}$. 

	We'll be done if we can show $p^{l+1} | p^{1 + 2(l-1)}$ and $p^{l + 1} | p^{p(l-1)}$, tha tis, that $1 + 2(l - 1) \geq l + 1$, and $l + 1 \leq p(l - 1)$, since $l \geq 2, p > 2$. Therefore, $B \equiv 0 \pmod{p^{l+1}}$.
\end{proof}

\begin{corollary}
	If $p \neq 2$ and $p \not \vert a \in \ZZ_{> 0}$, then $p^{l-1} \equiv \ord(1 + ap) \pmod{p^l}$.
\end{corollary}

We define $b \in \ZZ_{> 1}$, where $(b, n) = 1$, has order $l \pmod n$ if its order in $(\ZZ/n\ZZ)^*$ is $l$. Equivalently, $l$ is the least $\ZZ_{> 0}$ such that $b^l \equiv 1 \pmod n$. 

\section{Lecture 18}
Started the proof in lec 17, continue in lec 18.
\begin{proof}[Proof of \ref{17.1}]
	There exist primitive roots $\pmod p$. Pick one, call it $g \in \ZZ$. Then, $\overline{g} = \overline{g + p}$ so $g + p$ is also a primitive rood modulo $p$. If $g^{p-1} \equiv 1 \pmod{p^2}$, then $(g + p)^{p-1} \equiv g^{p-1} + (p - 1)g^{p-2} \cdot p \pmod{p^2}$.

	We see that the first term is equivalent to $1$. So, if $p$ does not divide $g^{p-2}$, then the $g^{p -1} + (p-1)g^{p-2} \cdot p \neq 1 \pmod{p^2}$, meaning $g$ is a primitive root mod $p$.

	Now, assume I can find a primitive root modulo $p$, call it $g'$. Then, I claim $g'$ is a primitive root mod $p^l$ for any $l > 0$.
\end{proof}

\section{Lecture 19}

\section{Lecture 20}
unsure. missed it. oops

\begin{definition}[Legendre Symbol] 
	$\frac{a}{p}$ is defined as the Legendre Symbol, which equals:
	\begin{itemize}
		\item $0$ if $a \equiv 0 \pmod p$
		\item $1$ if $a$ is another quadratic residue modulo $p$
		\item $-1$ if $a$ is a quadratic nonresidue modulo $p$.
	\end{itemize}
\end{definition}

\begin{lemma}[Gauss's Lemma] 
	Consider the set of least residues $S = \{ -(p-1)/2, -(p-3)/2, \ldots, -1, 1, 2, \ldots, (p-1)/2 \}$. Let $\mu$ be the number of least residues $a, 2a, \ldots, (p-1)a/2$ that are negative.

	Then, $(a / p) = (-1)^{\mu}$. 
\end{lemma}

\section{Lecture 21}
\subsection{Quadratic Reciprocity}
The question we're trying to answer is when primes are squares modulo $n$.

We introduced a set $S = \{-\frac{p-1}{2}, \ldots, -1, 1, \ldots, \frac{p-1}{2}\}$, which we call the \vocab{set of least residues} modulo $p$.

If $a\in\ZZ, p \not \vert a$, we form $\{a, 2a, \ldots, \frac{p-1}{2} a\}$ and we ask how many of this set are negative least residues mod $p$. Call this number $\mu$.

\begin{lemma}[Gauss' Lemma] 
	\[(-1)^\mu = (a / p), \text{ where } (p \nmid a).\]
\end{lemma}

\begin{proof}
	For each integer $1 \leq l \leq \frac{p-1}{2}$, let $\pm m_l$ be the least residue corresponding to $la$, where $m_l > 0$. As $l$ ranges over $1, 2, \ldots, \frac{p-1}{2}$, the number of negative signs appearing in the $\pm m_l$ is given by $\mu$.

	We claim now that $\left\{m_l | 1 \leq l \leq \frac{p-1}{2}\right\} = \{1, \ldots, (p-1)/2\}$. 

	We have \[\prod{l = 1}^{(p-1)/2} la = (-1)^\mu \prod_{i = 1}^{(p-1)/2} i.\] 
	This then equals \[a^{\frac{p-1}{2}} \left(\frac{p-1}{2}\right)! = (-1)^\mu \left(\frac{p-1}{2}\right)! \pmod p.\] 

	Plugging in the Legendre symbol gives us $\left(\frac{a}{p}\right) = a^{\frac{p-1}{2}} \equiv (-1)^\mu \pmod p$.
\end{proof}

\begin{proposition}
	$\left(\frac{2}{p}\right) \equiv (-1)^{\frac{p-1}{8}}$. 
\end{proposition}

\begin{proof}
	In this case $a = 2$, and $\mu$ is the number of numbers in the set $2, 2 \cdot 2, 3 \cdot 2, \ldots, 2\left(\frac{p-1}{2}\right)$
	which are strictly greater than $\frac{p-1}{2}$. Let $m$ be the integer characterized by 
	$2m \leq \frac{p-1}{2}$ and $2(m + 1) > \frac{p-1}{2}$. So $\mu = \frac{p-1}{2}= k$. So $\mu = \frac{p-1}{2} = m$.
\end{proof}

\begin{corollary}
	Infinitely many primes of the form $8k + 2$.
\end{corollary}

\begin{proof}
	Suppose not and $p_1, \ldots, p_n$ are all of them. Then, $n = (1 \cdot p_1, \cdot \cdots \cdot c_n)^2$, where $\lambda \equiv 0 \pmod n$ and $ n \equiv 0 \pmod p$ if $p | n$.

	Moreover, $Z$ is also a square modulo any $p | n$. So if $p | n, p \equiv 1, 7 \pmod 8$. Then, $\frac{n}{2} \equiv 7 \pmod 8$. Not all the prime divisors of $\frac{n}{2}$ can be $\equiv 1 \pmod 8$, so there exists a prime $p \nmid n$ such tat $p \equiv 7 \pmod 8)$, where $p \neq p_i$. 

	To be continued...
\end{proof}

\begin{theorem}[Quadratic Reciprocity]
	\label{qr}
	\begin{enumerate}[label=(\alph*)] 
		\item $(-1 / p) = (-1)^{(p-1)/2}$
		\item $(2 / p) = (-1)^{(p^2-1)/8}$
		\item $(p/q)(q/p) = (-1)^{((p-1)/2)((q-1)/2)}$.
	\end{enumerate}

	If one of $p$ or $q$ is $\equiv 1 \pmod 4$, then $p$ is a quadratic residue mod $q$ iff $q$ is a residue mod $p$. 

	If both $p$ and $q$ are $\equiv 3 \pmod 4$, then $p$ is a residue iff $q$ is a nonresidue.
\end{theorem}

This further implies that if either $p$ or $q$ is equivalent to $1$ modulo $4$, then $(p/q) = (q/p)$. If both are $3$ modulo $4$, then $(p/q) = -(q/p)$. 

\section{Lecture 22}
\begin{theorem}
	\begin{enumerate}[label=(\roman*)] 
		\item If $q \equiv 1 \pmod 4$ then $q$ is a residue mod $p$ iff $p \equiv r \pmod q$ where $r$ is a quadratic residue mod $q$. 
		\item If $q \equiv 3 \pmod 4$, then $q$ is a quadratic residue mod $p$ iff $p \equiv \pm b^2 \pmod{4q}$ for some odd integer $b$ prime to $q$.
	\end{enumerate}
\end{theorem}

\begin{proof} 
	\begin{enumerate} 
		\item $(q / p) = (p / q)$.
		\item
			If $q \equiv 3 \pmod 4$, then $(q / p) = (-1)^{((p-1)/2)((q-1)/2)} (p/q)$.

			Assume that $p \equiv \pm b^2 \pmod{4q}$ where $b$ is odd and $(b, q) = 1$.

			If $p \equiv b^2 \pmod{4q}$, then $p \equiv b^2 \equiv 1 \pmod 4$, so $(q / p) \equiv (p / q)$, and $p \equiv b^2 \pmod q$ so $(p / q) = 1$. So, $(q / p) = (1)(1) = 1$.

			If $p \equiv -b^2 \pmod{4q}$, then $p \equiv 3 \pmod 4$, so $(q / p) = -(p / q)$. As $p \equiv -b^2 \equiv q$ as well, we see that $(-b^2 / q) = (-1 / q)(b / q)^2  -1$. Therefore, $(p / q) = (-b^2 / q) = -1$, so $(q / p) = (-1)(-1) = 1$.
	\end{enumerate}
\end{proof}

\begin{example} 
	For which primes is $a = 6$ a quadratic residue?
\end{example}

We have \[(6 / p) = (2 / p) (3 / p) = 1 \iff (2 / p) = 1, (3 / p) = 1 \text{ or } (2 / p) = -1, (3 / p) = -1.\]

For the first case, we have $p \equiv \pm 1 \pmod 8$ and $p \equiv \pm 1 \pmod{12}$, so $p \equiv \pm 1 \pmod{24}$.

For the second case, we have $p \equiv \pm 3 \pmod 8$ and $p \equiv \pm 5 \pmod{12}$. The first congruence becomes $p \equiv \mp 5 \pmod 8$, so we see once again that both are simultaneously true if $p \equiv \pm 5 \pmod 24$.

Therefore, primes $p$ such that $p \equiv \pm 1$ or $p \equiv \pm 5$.

\begin{theorem}
	Suppose $a$ is a nonsquare integer. Then, there are infinitely many primes for which $a$ is a quadratic nonresidue.
\end{theorem}

\begin{corollary} 
	If $a$ is a quadratic residue mod every prime, it is a square. 
\end{corollary}

\section{Lecture 23}
The main focus for today is to prove the below theorem:
\begin{theorem}
	If $n \in \ZZ$ is a nonsquare integer, then $n$ is a quadratic nonresidue for infinitely many primes.
\end{theorem}

\begin{definition}[Jacobi symbol] 
	If $b \in \ZZ_{>0}$ is an odd integer with prime factorization $b = p_1 \cdots p_n$ and $a \in \ZZ$, then the symbol $(a/b) := (a/p_1)(a/p_2)\ldots(a/p_n)$ where 
	everything on the RHS is a Legendre symbol.
\end{definition}
The Jacobi system essentially expands the Legendre symbol to composite denominators.

\begin{lemma}
	Let $r, s$ be odd integers. Then, $$(rs - 1)/ 2 \equiv (r-1)/2 + (s-1)/2 \pmod 2$$ and $$(r^2s^2 - 1) \equiv (r^2 - 1)/8 + (s^2 - 1)/ 8 \pmod 2.$$
\end{lemma}

\begin{proof}
	If $r, s$ are odd, then $(r - 1)(s - 1) \equiv 0 \pmod 4$. Therefore, $rs - 1 \equiv (r-1) + (s-1) \pmod 4$. Dividing by $2$ gives us that $(rs - 1)/2 \equiv (r - 1)/2 + (s - 1)/2 \pmod 2$. \qed

	We observe that $4 | r^2 - 1$. Therefore, 
	\begin{align*}
		(r^2 - 1)(s^2 - 1) &\equiv 0 \pmod {16} \\
		\implies r^2s^2 - r^2 - s^2 + 1 &\equiv 0 \pmod {16} \\
		\implies r^2s^2 - 1 &\equiv (r^2 - 1) + (s^2 - 1) \pmod{16}.
	\end{align*}

	Dividing by $8$ gives us $(r^2s^2 - 1)/8 \equiv (r^2 - 1)/8 + (s^2 - 1)/8 \pmod 2$.
\end{proof}

\begin{corollary}
	If $r_1, \ldots, r_n$ are odd integers, then 
	\[\sum_{i = 1}^n (r_i - 1)/2 \equiv (r_1\cdots r_n - 1)/2 \pmod 2\] and 
	\[\sum_{i = 1}^n (r_i^2 - 1)/2 \equiv (r_1^2\cdots r_n^2 - 1)/ 8 \pmod 2.\] 
\end{corollary}

\begin{proof}
	Induct on $n$, $n = 2$ by the above lemma. Assume the result for $n$ and then for $n+1$ we have 
	\begin{align*}
		\sum_{i = 1}^{n+1} (r_i - 1)/2 &= \sum_{i=1}^n (r_i - 1)/2 + (r_{n+1}-1)/2 \\
									   &= (r_1 \cdots r_n - 1)/2 + (r_{n+1}-1)/2 \\
									   \equiv (r_1 \cdots r_{n+1} - 1)/2 \pmod 2.
	\end{align*}
\end{proof}

\begin{proposition}
	\begin{enumerate}[label=(\roman*)] 
		\item $(-1 / b) = (-1)^{(b-1)/2}$ 
		\item $(2 / b) = (-1)^{(b^2 - 1)/8}$
		\item $(a / b_1b_2) = (a/b_1)(a/b_2)$.
	\end{enumerate}
\end{proposition}

We prove this proposition somehow, then use the proof of that to prove the theorem 

\section{Lecture 24}
\begin{theorem}[Quadratic Reciprocity] 
	If $p, q$ are odd primes then \[(p/q)(q/p) = (-1)^{((p-1)/2 \cdot (q-1)/2))}.\]
\end{theorem}

Our proof will use the roots of unity. 

\section{Lecture 25} 
\subsection{Algebraic Numbers}
\begin{definition} 
	A complex number $s$ is an \vocab{algebraic number} if it satisfies a polynomial equation with rational coefficient. 

	It is an \vocab{algebraic integer} if it satisfies a monic polynomial equation with rational coefficients.

	These are represented as $\bar{\QQ}, \bar{\ZZ}$, respectively.
\end{definition}

If a number is not algebraic, then it is \vocab{transcendental}. 
\section{Lecture 26}
\subsection{Rings, Fields}
\begin{definition}[Ring] 
	A \vocab{ring} $R$ is a set $R$ closed under binary operations $+$ and $\cdot$, with the property that 
	\begin{enumerate}
		\item $(R, +)$ is an abelian group. 
		\item Multiplication is associative: $(a \cdot b) \cdot c = a \cdot (b \cdot c)$ 
		\item Multiplication distributes over addition: $a \cdot (b + c) = a \cdot b + a \cdot c$.
	\end{enumerate}
\end{definition}

\begin{definition}[Field]
	A commutative ring, where every nonzero element has a multiplicative inverse is called a \vocab{field}.
\end{definition}

\begin{proposition}
	$\overline{\ZZ}$ forms a ring. 
\end{proposition}

\begin{definition}
	A finite $\ZZ$-submodule of $\CC$ is a subset $\mathbf{V}$ of $\CC$ with the property that 
	\begin{enumerate} 
		\item $v + w \in \mathbf{V}$ if $v, w \in \mathbf{V}$.
		\item There exist $v_1, \ldots, v_n \in \mathbf{V} \subset \CC$ so that $\mathbf{V} = \{a_1v_1 + \cdots + a_nv_n \mid a_i \in \ZZ\}$.
	\end{enumerate}
\end{definition}

\begin{lemma}
	If we have a finite $\ZZ$-submodule $\mathbf{V}$ of $\CC$, and $\alpha \in \CC$ such that $\alpha \mathbf{V} = \mathbf{V}$, then $\alpha$ is an algebraic integer.
\end{lemma}

\begin{proof} 
	$\alpha$ on $\mathbf{V}$ is given an $n \times n$ matrix with integer coefficients.
\end{proof}

\begin{proposition}
	If $\alpha$ is an algebraic number, then $\alpha$ is a root of an irreducible unique monic polynomial $g(x) \in \QQ[x]$. If $h(x) \in \QQ[x]$ has $h(\alpha) = 0$, then $g(n) \mid h(x)$. 
\end{proposition}

\begin{proof}
	Here, we assume that $g(x)$ is irreducible. If $g(x) \nmid h(x)$, then $(g(x), h(x)) = 1$. Then, it follows that $r(x)g(x) + s(x)h(x) = 1$ for some $r, s$ and any $x$. However, for $x = \alpha$, we have $g(\alpha) = h(\alpha) = 0$, so this cannot be true. Therefore, $g(x) \mid h(x)$. 
\end{proof}

Therefore, the $g(x)$ so described is called the \vocab{minimal polynomial} of $\alpha$.

\begin{definition} 
	If $\alpha \in \CC$, let $\QQ(x)$ denote the smallest subfield of $\CC$ containing $\QQ$ and $\alpha$.
\end{definition}

Then $\QQ(x) = \left\{ \frac{g(x)}{h(x)} \mid g(x), h(x) \in \QQ[x], h(x) \in 0\right\}$. $g(x), h(x)$ are polynomials, and $g(x)$ is obtained by substituting $\alpha$ for $x$.

\section{Lecture 27} 
\subsection{Congruence} 
If $\omega_1, \omega_2, \gamma \in \ZZ$, we say $\omega_1 \equiv \omega_2 \pmod \gamma$ if $(\omega_1 - \omega_2) = \alpha \gamma$, with $\alpha \in \bar{\ZZ}$.

\begin{lemma}
	If $p$ is a prime, $\omega_1, \omega_2 \in \ZZ$, then $(\omega_1 + \omega_2)^p \equiv \omega_1^p + \omega_2^p \pmod p$.
\end{lemma}

\begin{proof}
	This is essentially a proof that all of the middle terms are divisible by $p$.
	$(\omega_1 + \omega_2)^p$ can be expanded via the binomial theorem, to get 
	that $\left( \frac{p}{n} \right)$ and $p \mid \dbinom{p}{n}$ if $1 \leq n \leq p$.

	A proof of the quadratic character of $2$ via Gaussian sums:
\end{proof}

\begin{definition}
	$g_a = \sum_{t = 0}^{p-1} \left(\frac{t}{p}\right) \zeta^{at}$ is an example of a quadratic Gauss sum.
\end{definition}

\begin{lemma}
	h
\end{lemma}

\begin{corollary}
	\[p^{-1} \sum_{t = 0}^{p - 1} \zeta^{t(x - y)} = \delta(x, y) = \begin{cases} 1 & x \not \equiv y \pmod p \\ 0 & x \equiv y \pmod p \end{cases}.\] 
\end{corollary} 

\end{document}
