\documentclass{article}
\usepackage[utf8]{inputenc}
\usepackage{fancyhdr}
\usepackage{graphicx}
\usepackage{dirtytalk}
\usepackage[portrait, margin=0.5in, top=0.8in, bottom=0.3in]{geometry}
\usepackage{amsmath, amssymb}
\setlength{\parskip}{1em}
\setlength{\parindent}{0cm} 
\usepackage[mdthm,simplethm]{test}
\usepackage[utf8]{inputenc}
\usepackage{float}
\usepackage{physics}
\usepackage{sectsty}
\usepackage{tikz}
\usepackage{circuitikz}
\allsectionsfont{\normalfont\sffamily\bfseries}

\usepackage{hyperref}
\usepackage{nameref}
\hypersetup{
    colorlinks=true,
    linkcolor=blue,
    filecolor=magenta,      
    urlcolor=cyan,
    pdftitle={Overleaf Example},
    pdfpagemode=FullScreen,
    }

\urlstyle{same}

\setlength{\parindent}{0pt}
\pagestyle{fancy}
\lhead{Math 115 - Introduction to Number Theory}
\rhead{Albert Ye}

\title{Common Core 5th Grade Curriculum}
\author{albert ye}
\date{\today}

\begin{document}
\maketitle

\section{Lecture 1}
\begin{definition}
    An integer $p \neq 0, 1, -1$ is \vocab{prime} if the only integers which divide $p$ are $\pm 1$ and $\pm p$.
\end{definition}

Recall that the integers $\ZZ = \{\ldots, -3, -2, -1, 0, 1, 2, 3, \ldots\}$, $\NN = \{0, 1, 2, 3, \ldots\}$.

\begin{theorem}[Twin Prime Conjecture]
    There are infinitely many $p \in \NN$ such that $p$ is prime and $p + 2$ is prime.
\end{theorem}

Yitang Zhang proved bounded gaps between primes, so there are infinitely many prime $p, p + N$.

\begin{theorem}[Goldbach Conjecture]
    Every even number can be written as the sum of two primes.
\end{theorem}

Vinagradar proved that every odd number can be written as the sum of $3$ primes. The proof should use something called sieves.

\begin{proposition}
    There are infinitely many primes.
\end{proposition}

\begin{proof}
    Suppose not and $p_1, \ldots, p_n$ are all the primes. Then, let $p_1\cdots p_n + 1 = N$. 
    
    As we will see, every integer admits a unique decomposition into a product of primes.
\end{proof}

\subsection{Counting Primes}
Let $\pi(x) : N \to \NN$ return the number of primes $p$ such that $0 < p < x$.

Then, $\pi(x)$ is unbounded: $\lim_{x \to \infty} \pi(x) = \infty$. 

\begin{theorem}[Prime Number Theorem]
    \[\lim \frac{\pi(x)}{x / \log x} = 1.\] In other words, $\pi(x) \to \frac{x}{\log x}$>
\end{theorem}

A better approximation is $\mathrm{Li}(x) = \int_{2}^x \frac{dt}{\log t}$. The error for $\mathrm{Li}(x)$ is $|\pi(x) - \mathrm{Li}(x)| = O(\log x \sqrt x)$.

\subsection{Prime Factorization}
\begin{theorem}[Uniqueness of Prime Factorization]
    \label{uniq}
    Every integer $0 \neq n \in \ZZ$ can be written as \[n = (-1)^{Z(n)} \prod_{p \text{ prime}} p^{a_p} \qquad a_p \in \NN,\] where all but finitely many $a_p$ are zero, $\epsilon(n) = \begin{cases} 0 & n > 0 \\ 1 & n < 0\end{cases}$.
\end{theorem}

To prove this, we first look at a lemma:
\begin{lemma}
    \label{lemma:1}
    If $a, b \in \ZZ$ and $b > 0$, there exist integers $q, r$ such that $a = qb + r$ and $0 \leq r < b$.
\end{lemma}

\begin{proof}
    Consider the set of integers of the form $\{a - xb | x \in \ZZ\} = S$. The set $S$ contains infinitely many positive integers, so contains a least positive integer $r = a - qb$.

    \begin{remark}
        This property does not hold for $S \subset \QQ$. Consider $S = \{1, \frac{1}{2}, \frac{1}{4}, \ldots\}$. 
    \end{remark}
\end{proof}

The rest of the proof will follow later.

\begin{definition}
    \label{multset}
    Let $a_1, \ldots, a_n$ be integers. Denote $(a_1, \ldots, a_n)$ to be the set $\{b_1a_1 + \cdots + b_na_n | b_i \in \ZZ\}$.
\end{definition}

\section{Lecture 2}
\subsection{Prime Factorization, cont.}
Recall the theorem of uniqueness of prime factorizations. Also recall that a prime number $p$ is an integer $\neq 0$, so that the only divisors of $p$ are $\pm 1$ and $\pm p$.

\begin{definition}
    If $0 \neq a \in \ZZ$ and $p \in \ZZ$ is prime, let $\ord_p a$ denote the largest integer $n$ such that $p^n \vert a$, i.e. $a = p^n b$.

    We define $\ord_p 0 = \infty$.
\end{definition}

\begin{lemma}
    \label{lemma:2}
    If $a, b \in \ZZ$, then there exists $d \in \ZZ$ such that $(d) = (a, b)$. Recall Definition \ref{multset} for $(a_1, a_2, \ldots, a_n)$.
\end{lemma}


\begin{proof}
    Let $d$ be the smallest integer $> 0$ in $(a, b)$. We claim that $(d) = (a, b)$. As $d \in (a, b)$, we see that $(d) \subseteq (a, b)$. We have to show that $(a, b) \subseteq (d)$.

    Take $c \in (a, b)$, then we see from  \ref{lemma:1} that $c = qd + r$ with $0 \leq r < d$. THen $r = c-qd \in (a,b)$. By minimality of $d$, we see that $r = 0$, so $c = qd$ implie $c \in (d)$.
\end{proof}

\begin{definition}
    If $a, b \in \ZZ$, then a greatest common divisor $d$ of $a, b$ is an integer which divides $a, b$ such that any other integer $c$ with that property satisfies $c | d$.
\end{definition}

\begin{remark}
    If we insist $d \geq 0$, then it is unique. Because if $c, d \geq 0$ are both $\gcd(a, b)$, then $c | d$ and $d | c$, which implies $c = \pm d$, but because of positivity we must have $c = d$.
\end{remark}

\begin{proposition}
    If $a, b \in \ZZ$, then the $d$ appearing in \ref{lemma:2} s.t. $d = (a, b)$ is a greatest common divisor of $a, b$.
\end{proposition}

\begin{proof}
    If $(d) = (a, b)$, then $a \in (d) = d\ZZ \implies d | a$. If $c \in \ZZ$ is any common divisor of $a$ and $b$, then $c$ divides $an + bm$ for all $m, n \in \ZZ$. As $d \in (a, b)$, $d$ has this form, so $c | d$.

    Thus, by definition, $d$ must be the greatest common divisor.
\end{proof}

\begin{definition}
    We say that $a, b \in \ZZ$ are \vocab{relatively prime} if $(a, b) = 1$.

    In other words, the only nonzero integers that divide $a$ and $b$ are $\pm 1$.
\end{definition}

\begin{lemma}
    \label{lemma:3}
    Suppose $a | bc$, and $(a, b) = 1$. Then, $a | c$.
\end{lemma}

\begin{proof}
    $(a, b) = 1$ implies $1 = an + bm$ for some $n, m$. So $c = acn + bcm$. Notice that the right term contains $bc$ and the left term contains $a$, so $c$ must be divisible by $a$.
\end{proof}

\begin{corollary}
    \label{cor:3}
    If $p$ is prime and $p | ab$, then $p | a$ or $p | b$.
\end{corollary}

\begin{proof}
    If $(p, a) = p$, then we're done as $p | a$.

    Suppose instead that $(p, a) = 1$. From \ref{lemma:3}, we have $p | b$.
\end{proof}

We take the contrapositive to see that if a prime $p$ doesn't divide $a$ or $b$, then it doesn't divide $ab$.

\begin{proposition}
    Fix a prime $p$. If $a, b \in \ZZ$, then $\ord_p ab = \ord_p a + \ord_p b$.
\end{proposition}

\begin{proof}
    Let $\ord_p a = n, \ord_p b = m$. Then, we see that $a = p^nc, b = p^md$ where $p \not \vert c, p \not \vert d$. So $ab = p^nc \cdot p^md = p^{n+m}(cd)$. We know that $p$ cannot divide $cd$ from \ref{cor:3}, so $\ord_p ab = n + m$.
\end{proof}

Now, we can finally prove Theorem \ref{uniq}.
\begin{proof}[Proof of \ref{uniq}]
    Fix $n \in \ZZ$ and suppose that $n = (-1)^{\epsilon(n)} \prod_{p} p^{a_p}$.

    Then, fix a prime $q$. We see that \[\ord_{q} n = 0 + \sum_{p}a_p \ord_q p = a_q.\] This is because $\ord_q p = \begin{cases}1 & q = p \\ 0 & q \neq p\end{cases}$. This implies that the only factors that will contribute to $\ord_q n$ are the terms of $q$, of which there are $a_q$.

    Hence, $a_p$ for each prime $p$ is determined solely by $n$, so the prime factorization is unique.
\end{proof}

\section{Lecture 3}

\begin{lemma}
    \label{lemma:3:1}
    Every nonconstant irreducible polynomial has a factorization into nonconstant irreducible polynomials.
\end{lemma}

\section{Lecture 4}
\subsection{Factorization of Polynomials}
Recall \ref{lemma:3:1} from last lecture.

Again let $k = \QQ, \RR, \CC$. 

\begin{definition}
    A nonzero polynomial is called \vocab{monic} if the coefficient of its leading term is $1$.
\end{definition}

\begin{definition}
    If $p(x) \in k[x]$ is nonconstant irreducible, and $0 \neq q(x) \in k[x]$ is any other polynomial. Let $\ord_p q$ be defined as the greatest integer $n \geq 0$ such that $p^n(x) \vert g(x)$ but $p^{n+1}(x) \not \vert g(x)$.
\end{definition}

\begin{theorem}
    \label{polyfac}
    Every nonconstant polynomial $g(x)$ admits a unique factorization of the form $g(x) = c \prod_{p(x)} p(x)^{a_p}$, where $c \in k^x = k \setminus \{0\}$ and the product is over all irreducible, nonconstant, monic polynomials.

    Then, $a_p = \ord_p g$, and $c$ is the leading term of $g$.
\end{theorem}

We start with the following lemma: 

\begin{lemma}
    If $f(x), g(x) \in k[x]$ are polynomials with $0 \neq g(x)$ then we can find polynomials $q(x)$ and $r(x)$ with either $r(x) = 0$ or $0 \leq \deg r(x) < \deg g(x)$ s.t. $f(x) = q(x)g(x) + r(x)$.
\end{lemma}

\begin{proof}
    If $g|f$, then $g(x)q(x) = f(x)$ for some $q(x)$, and let $r(x) = 0$. Suppose otherwise, and $f \neq 0$. Consider the set $f(x) \in \{f(x) - h(x)g(x), h(x) \in k[x]\}$, and let $q(x)$ be such that $r(x) = f(x) - q(x)g(x)$ is of least degree in this set. 

    It remains to show $r = 0$ or $\deg r < \deg g$. Suppose otherwise, and that $r(x)$ has leading term $ax^d$ and $g(x)$ has leading term $bx^n$ with $d \geq n$. Let $m9x) = \frac{a}{b} x^{d - n} g(x)$. Then $m(x)$ is a polynomial such that $\deg(r(x) - m(x)) < \deg r(x)$. 

    However, $r(x) - m(x) = f(x) - (q(x) + \frac{a}{b} x^{d - n}) g(x)$, so $r(x) - m(x) \in S$. This contradicts the definitions of $r(x)$.
\end{proof}

\begin{definition}
    If $f_1(x), \ldots, f_n(x)$ are polynomials, let $(f_1, f_2, \ldots, f_n)$ be defined similarly to integers.
\end{definition}

\begin{lemma}
    Given $f(x), g(x) \in k(x)$, there is a $d(x) \in k[x]$ s.t. $(f, g) = (d)$. 
\end{lemma}

\begin{proof}
    Let $d(x)$ be a polynomial of least degree in $(f, g)$. We have $(d) \subset (f, g)$. Let $c(x) \in (f, g)$. Then, if $d \vert c$, we're done. If not, then there exists $q(x), r(x)$ s.t. $c(x) = q(x)d(x) + r(x)$, with $\deg r(x) < \deg d(x)$. Then $r(x) = c(x) - q(x)d(x) \in (f, g)$, which is a contradiction as $\deg r < \deg d$. 
\end{proof}

\section{Lecture 5}
Continue proving \ref{polyfac}.

\begin{definition}
    We say $f(x), g(x) \in k[x]$ are \vocab{relatively prime} if $(f, g) = 1$.
\end{definition}

\begin{definition} 
    A greatest common divisor, or $\gcd$ of $f$ and $g \in k[x]$ is a polynomial $d(x)$ which divides $f$ and $g$ and has the property that if $c(x) \in k[x]$ divides $f$ and $g$ then $c | d$. (Ambiguous up to a scalar.)
\end{definition}

\begin{lemma}
    \label{poly:div}
    If $f$ and $g$ are relatively prime and $f | gh$, then $f | h$. 
\end{lemma}

\begin{proof}
    If $(f, g) = 1$ then $1 = a(x)f(x) + b(x)g(x)$. So $h(x) = a(x)f(x)h(x) + b(x)g(x)h(x) = f(x)(a(x)h(x) + b(x)j(x))$ for some other polynomial $j(x)$. Then, $f(x) | h(x)$.
\end{proof}

If $d(x) = (f(x), g(x))$ and $x \in k^x$ then $\alpha d$ is also a gcd o $f$ and $g$; $(\alpha d) = (d)$. 

Now, recall that a nonconstant polynomial $f(x)$ is \vocab{irreducible} if its only divisors are of the form $\alpha f$ or $\alpha$ ($\alpha \in k^*$); i.e. if any polynomial divides $f$, it's either a scalar or a scalar multiple of $f$.

\begin{lemma}
    If $p(x)$ is irreducible and $p | fg$, then $p | f$ or $p | g$. 
\end{lemma}

\begin{proof}
    $(p, f) = (1)$ or $(p) = (\alpha p)$ for all $x \in k^*$. If $(p, f) = (p)$, then $p | f$. Otherwise, $(p, f) = (1)$, so from Lemma \ref{poly:div} we have $p | g$. 
\end{proof}

\begin{definition}[Order in Polynomial Terms]
    If $p$ is a nonconstant polynomial and $g \neq f \in k[x]$ then $\ord_p f$ is the largest $a \in \ZZ_{\geq 0}$ such that $p^a | f$. 
\end{definition}

\begin{lemma}
    If $p(x) \in k[x]$ is irreducible and $a, b \in k[x]$, then $\ord_p(ab) = \ord_p(a) + \ord_p(b)$. 
\end{lemma} 

Finally, we can prove \ref{polyfac}. 

\begin{proof}
    Weite $0 \neq f(x) = c \prod_p p(x)^{a_p}$. For every monic irreducible polynomial $q$, $\ord_q f = \sum_f a_p \ord_q p$, and we see that $\ord_q p = \begin{cases}1 & q = p \\ 0 & q \neq p.\end{cases}$. This must be $a_q$.

    The scalar $c$ is the leading coefficient of $f$, so every polynomial factorization uniquely determines one polynomial. 
\end{proof}

\section{Lecture 6}
\begin{proposition}
    If $k = \QQ, \RR, \CC$ (any field) then $k[x]$ contains infinitely many irreducible polynomials.
\end{proposition}

\begin{proof}
    Suppose not, and $p_1(x), \ldots, p_n(x)$ exhaust the irreducible polynomials. Thus $q(x) = 1 + p_1(x)p_2(x) \cdots p_n(x)$ is a polynomial not divisible by the $p_i(x)$, but it must factor into a product of the $p_i(x)$, a contradiction. 
\end{proof}

\begin{lemma}
    Every integer $n \neq 0$ can be written as $n = ab^2$ where $a$ is squarefree.
\end{lemma}

\begin{definition}
    An integer $n \neq 0$ is squarefree if it isn't divisible by the square of any prime.
\end{definition}

\begin{proof}
    If $|n| = 1$ then it's squarefree. If $|n| > 1$ then $n = (-1)^{\epsilon(n)} p_1^{2a_1 + b_1} \cdots p_m^{2a_m + b_m}$, where $b_i$ is either $0$ or $1$ for all $i$. Then, in turn, \[n = [p_1^{2a_1}\cdots p_m^{2a_m}][(-1)^{\epsilon(n)} p_1^{b_1}\cdots p_m^{b_m}].\] We see that the first term is $b^2$ and the second term is a squarefree $a$.
\end{proof}

\begin{definition}
    $\nu(n) = $number of positive divisors 

    $\sigma(n) = $sum of positive divisors 
\end{definition}

\begin{proposition}
    Let $n \in \ZZ_{> 1}$ have a prime factorization $n = p_1^{a_1} \cdots p_m^{a_m}$. Then,
    \begin{itemize}
        \item $\nu(n) = (a_1 + 1)(a_2 + 1)\cdots(a_n + 1)$
        \item $\sigma(n) = \left(\sum_{i = 0}^{a_1} p_1^i\right) \cdots \left(\sum_{i = 0}^{a_n} p_n^i\right)$.
    \end{itemize}
\end{proposition}

Recall that $\sum_{n = a}^b x^n = \frac{x^{b + 1} - x^a}{x - 1}$, so $\sigma(n) = \left(\frac{p_1^{a_1 + 1} - 1}{p_1 - 1}\right) \cdots \left(\frac{p_n^{a_n + 1} - 1}{p_n - 1}\right)$.

\begin{definition}
    An integer $> 0$ is \vocab{perfect} if $\sigma(n) = 2n$.
\end{definition}

Euler claimed that every even perfect number can be written as $2^m(2^{m + 1} - 1)$, where $2^{m + 1} - 1$ is a Mersenne prime.

\begin{definition}[Mobius Mu Function]
    The Mobius $\mu : Z_{> 0} \to \{0, \pm 1\}$ returns $\mu(n) = 0$ if $n$ is not squarefree, $\mu(1) = 1$, and if $n > 1$, $n = p_1, \ldots, p_m$, then $\mu(n) = (-1)^m$.
\end{definition}

\begin{proposition}
    If $n > 1$ then $\sum_{d | n} \mu(d) = 0$.
\end{proposition}

\begin{proof}
    $n = p_1^{a_1} \cdots p_m^{a_m}$. Notice that for any $a_i > 1$, we can ignore and take mod $2$ because non-squarefree implies a Mobius of $0$.
    
    Therefore, $\sum_{d|n} \mu(d) = \sum \mu(p_1^{\epsilon_1} \cdots p_m^{\epsilon_m}) = (1 - 1)^m = 0$.
\end{proof}

\begin{definition}
    If $f, g$ are two functions $\ZZ_{> 0} \to \CC$, then the Dirichlet convolution of $f$ and $g$ is defined to be $(f \cdot g)(n) = \sum_{d | n} f(d)g(\frac{n}{d})$.
\end{definition}

\begin{remark}
    Dirichlet convolution is associative; given $f, g, h : \ZZ_{> 0} \to \CC$, then $((f \cdot g) \cdot h)(n) = (f \cdot (g \cdot h))(n) = \sum f(d_1) g(d_2) h(d_3)$,
\end{remark}

\begin{definition}
    Let $1(n) = \begin{cases}1 & n = 1 \\ 0 & n > 1\end{cases}$. Then, $(f * 1)(N) = \sum_{d | n} f(d)$. 
\end{definition}

\begin{theorem}[Mobius Inversion]
    If $f : \ZZ_{> 0} \to \CC$ and $F(n) = \sum_{d | n) f(d)}$, then $\sum_{d|n} F(d) \mu\left(\frac{n}{d}\right) = f(n)$, or as we simplify it, $\mu \times F = f$.
\end{theorem}

\section{Lecture 7}
\begin{definition}[Euler Totient]
    We define $\phi: \ZZ_{> 0} \to \ZZ_{> 0}$. $\phi(n)$ is the number of integers in $[1, n]$ relatively prime to $n$. 

    $\phi(1) = 1$, $\phi(p) = p-1$ for prime $p$.
\end{definition}

\begin{proposition}
    $(\phi \cdot )(n) = \sum_{d | n} \phi(d) = n$.
\end{proposition}

\begin{proof}
    Consider the set $\left\{\frac{1}{n}, \frac{2}{n}, \ldots, \frac{n}{n}\right\}$. Write these fractions in lowest terms. 
    
    For each $d | n$, we wish to count the functions above with $d$ in lowest terms. These fractions will be a subset of the fractions $\frac{a}{n}$ where $\frac{n}{d} | a$, i.e. a subset of the fractions $\left\{\frac{1}{d}, \frac{2}{d}, \ldots, \frac{d}{d}\right\}$. There are $\phi(d)$ many fractions on this list with $d$ in the domain, when written in lowest terms.

    So if $J_d \subset \left\{\frac{1}{n}, \frac{2}{n}, \ldots, \frac{n}{n}\right\}$ corresponds to the fractions of denominator $d$ in lowest terms, then $S = \bigcup_{d | n} J_d$, and $n = |S| = \sum_{d | n} |J_d| = \sum_{d | n} \phi(d)$. 
\end{proof}

From Mobius inversion, we have $\phi = (\phi \cdot 1) \cdot \mu$. We know that $(\phi \cdot 1) = id$ where $id(n) = n$, so we have $\mu \cdot id = \sum_{d | n} \mu(d) \frac{n}{d}$. Now, let $n = p_1^{a_1}\cdots p_m^{a_m}$.  Then,
\begin{align*}
    \mu \cdot id &= n - \sum_{i} \frac{n}{p_i} + \sum_{i < j} \frac{n}{p_ip_j} - \sum_{i < j < k} \frac{n}{p_ip_jp_k} \cdots \text{ (by definition of Mobius inversion)}\\
    &= n\left(1 - \frac{1}{p_1}\right)\left(1 - \frac{1}{p_2}\right)\cdots \left(1 - \frac{1}{p_m}\right) = \phi(n).
\end{align*}

\begin{theorem}
    $\sum_{p \text{ prime}} \frac{1}{p}$ diverges.
\end{theorem}

\begin{proof}
    Of $n \in \ZZ_{> 0}$, let $p_1, \ldots, p_{\pi(n)}$ be the primes $\leq n$ and let \[\lambda(n) = \prod_{i = 1}^{\pi(n)} \left(1 - \frac{1}{p_i}^{-1}\right).\] Notice that each inner value for the product term is $\sum_{a = 0}^\infty \left(\frac{1}{p_i}\right)^a$. 

    Then, $\lambda(n) = \sum \frac{1}{p_1^{a_1} \cdots p_{\pi(n)}^a}$, where the sum is over all $\pi(n)$-tuples $(a_1, \ldots, a_{\pi(n)}) \in \ZZ_{\geq 0}^{\pi(n)}$.

    Now, we claim $\lambda(n) \to \infty$ as $1 + \frac{1}{2} + \frac{1}{3} + \cdots + \frac{1}{n} < \lambda(n)$.

    "I'll pick it up next time" -Owen Barrett
\end{proof}
\end{document}