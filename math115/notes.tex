\documentclass{article}
\usepackage[utf8]{inputenc}
\usepackage{fancyhdr}
\usepackage{graphicx}
\usepackage{dirtytalk}
\usepackage[portrait, margin=0.5in, top=0.8in, bottom=0.3in]{geometry}
\usepackage{amsmath, amssymb}
\setlength{\parskip}{1em}
\setlength{\parindent}{0cm} 
\usepackage[mdthm,simplethm]{test}
\usepackage[utf8]{inputenc}
\usepackage{float}
\usepackage{physics}
\usepackage{sectsty}
\usepackage{tikz}
\usepackage{circuitikz}
\allsectionsfont{\normalfont\sffamily\bfseries}

\usepackage{hyperref}
\usepackage{nameref}
\hypersetup{
    colorlinks=true,
    linkcolor=blue,
    filecolor=magenta,      
    urlcolor=cyan,
    pdftitle={Overleaf Example},
    pdfpagemode=FullScreen,
    }

\urlstyle{same}

\setlength{\parindent}{0pt}
\pagestyle{fancy}
\lhead{Math 115 - Introduction to Number Theory}
\rhead{Albert Ye}

\title{Common Core 5th Grade Curriculum}
\author{albert ye}
\date{\today}

\begin{document}
\maketitle

\section{Lecture 1}
\begin{definition}
    An integer $p \neq 0, 1, -1$ is \vocab{prime} if the only integers which divide $p$ are $\pm 1$ and $\pm p$.
\end{definition}

Recall that the integers $\ZZ = \{\ldots, -3, -2, -1, 0, 1, 2, 3, \ldots\}$, $\NN = \{0, 1, 2, 3, \ldots\}$.

\begin{theorem}[Twin Prime Conjecture]
    There are infinitely many $p \in \NN$ such that $p$ is prime and $p + 2$ is prime.
\end{theorem}

Yitang Zhang proved bounded gaps between primes, so there are infinitely many prime $p, p + N$.

\begin{theorem}[Goldbach Conjecture]
    Every even number can be written as the sum of two primes.
\end{theorem}

Vinagradar proved that every odd number can be written as the sum of $3$ primes. The proof should use something called sieves.

\begin{proposition}
    There are infinitely many primes.
\end{proposition}

\begin{proof}
    Suppose not and $p_1, \ldots, p_n$ are all the primes. Then, let $p_1\cdots p_n + 1 = N$. 
    
    As we will see, every integer admits a unique decomposition into a product of primes.
\end{proof}

\subsection{Counting Primes}
Let $\pi(x) : N \to \NN$ return the number of primes $p$ such that $0 < p < x$.

Then, $\pi(x)$ is unbounded: $\lim_{x \to \infty} \pi(x) = \infty$. 

\begin{theorem}[Prime Number Theorem]
    \[\lim \frac{\pi(x)}{x / \log x} = 1.\] In other words, $\pi(x) \to \frac{x}{\log x}$>
\end{theorem}

A better approximation is $\mathrm{Li}(x) = \int_{2}^x \frac{dt}{\log t}$. The error for $\mathrm{Li}(x)$ is $|\pi(x) - \mathrm{Li}(x)| = O(\log x \sqrt x)$.

\subsection{Prime Factorization}
\begin{theorem}[Uniqueness of Prime Factorization]
    \label{uniq}
    Every integer $0 \neq n \in \ZZ$ can be written as \[n = (-1)^{Z(n)} \prod_{p \text{ prime}} p^{a_p} \qquad a_p \in \NN,\] where all but finitely many $a_p$ are zero, $\epsilon(n) = \begin{cases} 0 & n > 0 \\ 1 & n < 0\end{cases}$.
\end{theorem}

To prove this, we first look at a lemma:
\begin{lemma}
    \label{lemma:1}
    If $a, b \in \ZZ$ and $b > 0$, there exist integers $q, r$ such that $a = qb + r$ and $0 \leq r < b$.
\end{lemma}

\begin{proof}
    Consider the set of integers of the form $\{a - xb | x \in \ZZ\} = S$. The set $S$ contains infinitely many positive integers, so contains a least positive integer $r = a - qb$.

    \begin{remark}
        This property does not hold for $S \subset \QQ$. Consider $S = \{1, \frac{1}{2}, \frac{1}{4}, \ldots\}$. 
    \end{remark}
\end{proof}

The rest of the proof will follow later.

\begin{definition}
    \label{multset}
    Let $a_1, \ldots, a_n$ be integers. Denote $(a_1, \ldots, a_n)$ to be the set $\{b_1a_1 + \cdots + b_na_n | b_i \in \ZZ\}$.
\end{definition}

\section{Lecture 2}
\subsection{Prime Factorization, cont.}
Recall the theorem of uniqueness of prime factorizations. Also recall that a prime number $p$ is an integer $\neq 0$, so that the only divisors of $p$ are $\pm 1$ and $\pm p$.

\begin{definition}
    If $0 \neq a \in \ZZ$ and $p \in \ZZ$ is prime, let $\ord_p a$ denote the largest integer $n$ such that $p^n \vert a$, i.e. $a = p^n b$.

    We define $\ord_p 0 = \infty$.
\end{definition}

\begin{lemma}
    \label{lemma:2}
    If $a, b \in \ZZ$, then there exists $d \in \ZZ$ such that $(d) = (a, b)$. Recall Definition \ref{multset} for $(a_1, a_2, \ldots, a_n)$.
\end{lemma}


\begin{proof}
    Let $d$ be the smallest integer $> 0$ in $(a, b)$. We claim that $(d) = (a, b)$. As $d \in (a, b)$, we see that $(d) \subseteq (a, b)$. We have to show that $(a, b) \subseteq (d)$.

    Take $c \in (a, b)$, then we see from  \ref{lemma:1} that $c = qd + r$ with $0 \leq r < d$. THen $r = c-qd \in (a,b)$. By minimality of $d$, we see that $r = 0$, so $c = qd$ implie $c \in (d)$.
\end{proof}

\begin{definition}
    If $a, b \in \ZZ$, then a greatest common divisor $d$ of $a, b$ is an integer which divides $a, b$ such that any other integer $c$ with that property satisfies $c | d$.
\end{definition}

\begin{remark}
    If we insist $d \geq 0$, then it is unique. Because if $c, d \geq 0$ are both $\gcd(a, b)$, then $c | d$ and $d | c$, which implies $c = \pm d$, but because of positivity we must have $c = d$.
\end{remark}

\begin{proposition}
    If $a, b \in \ZZ$, then the $d$ appearing in \ref{lemma:2} s.t. $d = (a, b)$ is a greatest common divisor of $a, b$.
\end{proposition}

\begin{proof}
    If $(d) = (a, b)$, then $a \in (d) = d\ZZ \implies d | a$. If $c \in \ZZ$ is any common divisor of $a$ and $b$, then $c$ divides $an + bm$ for all $m, n \in \ZZ$. As $d \in (a, b)$, $d$ has this form, so $c | d$.

    Thus, by definition, $d$ must be the greatest common divisor.
\end{proof}

\begin{definition}
    We say that $a, b \in \ZZ$ are \vocab{relatively prime} if $(a, b) = 1$.

    In other words, the only nonzero integers that divide $a$ and $b$ are $\pm 1$.
\end{definition}

\begin{lemma}
    \label{lemma:3}
    Suppose $a | bc$, and $(a, b) = 1$. Then, $a | c$.
\end{lemma}

\begin{proof}
    $(a, b) = 1$ implies $1 = an + bm$ for some $n, m$. So $c = acn + bcm$. Notice that the right term contains $bc$ and the left term contains $a$, so $c$ must be divisible by $a$.
\end{proof}

\begin{corollary}
    \label{cor:3}
    If $p$ is prime and $p | ab$, then $p | a$ or $p | b$.
\end{corollary}

\begin{proof}
    If $(p, a) = p$, then we're done as $p | a$.

    Suppose instead that $(p, a) = 1$. From \ref{lemma:3}, we have $p | b$.
\end{proof}

We take the contrapositive to see that if a prime $p$ doesn't divide $a$ or $b$, then it doesn't divide $ab$.

\begin{proposition}
    Fix a prime $p$. If $a, b \in \ZZ$, then $\ord_p ab = \ord_p a + \ord_p b$.
\end{proposition}

\begin{proof}
    Let $\ord_p a = n, \ord_p b = m$. Then, we see that $a = p^nc, b = p^md$ where $p \not \vert c, p \not \vert d$. So $ab = p^nc \cdot p^md = p^{n+m}(cd)$. We know that $p$ cannot divide $cd$ from \ref{cor:3}, so $\ord_p ab = n + m$.
\end{proof}

Now, we can finally prove Theorem \ref{uniq}.
\begin{proof}[Proof of \ref{uniq}]
    Fix $n \in \ZZ$ and suppose that $n = (-1)^{\epsilon(n)} \prod_{p} p^{a_p}$.

    Then, fix a prime $q$. We see that \[\ord_{q} n = 0 + \sum_{p}a_p \ord_q p = a_q.\] This is because $\ord_q p = \begin{cases}1 & q = p \\ 0 & q \neq p\end{cases}$. This implies that the only factors that will contribute to $\ord_q n$ are the terms of $q$, of which there are $a_q$.

    Hence, $a_p$ for each prime $p$ is determined solely by $n$, so the prime factorization is unique.
\end{proof}
\end{document}